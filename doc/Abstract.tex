\documentclass[a4paper,11pt]{article}
\usepackage[utf8]{inputenc}
 
%opening
\title{A General Purpose Local Search Solver }
\author{Bo Stentebjerg-Hansen}

\begin{document}

\maketitle

\section*{Description}
% Intro -> hvad er der lige nu
% Hvad skal jeg arbejde med
The field of optimization can be split into several subfields depending on the nature of the decision variables 
(continuous, discrete) and the structure of the problem (linear, non linear, combinatorial, convex, non 
convex). The main focus of this thesis is discrete optimization both linear and non linear. 
Several solvers are available for discrete optimization. The mixed integer linear programming (MILP) approach has 
solvers such as GLPK, Gurobi and CPLEX. There are also constraint programming (CP) solvers, such as Gecode. All these 
solvers solve the problem exactly, but for some problems thit is not always possible due to the computational 
cost. Another approach is to use local search and find a good solution fast by making a trade off between speed 
and solution quality. \medskip \\ 
There exists a vast literature about how to make good local search solvers for specific problems.
However, only a few attempts have been made to use local search for general purpose solvers like 
mathematical programming and constraint programming. Comet was a successful CP based solver and that allowed to 
use local search to find a good solution fast but the project is now abandoned. Currently LocalSolver is a commercial 
product that combines a local search solver with a problem modeling language. It can be used to solve mathematical, 
constraint, continuous, and non linear programming. \medskip \\
In this project, a general heuristic solver based on local search will be developed. Initially, it will be a solver 
for problems formulated as binary programming problems. For the part of 
software engineering we will combine ideas from Comet, LocalSolver, Gecode and Easylocal which is a local search 
framework.  \medskip \\
Beside the basic components of local search there are several elements that will be studied to improve the 
solver. One of these elements is preprocessing that can reduce the size of the search space before the local search is 
started and improve the speed of the solver considerably. Other elements that will be studied are invariants and 
directed acyclic graphs (DAGs) to represent efficiently the dependencies between the variables. The choice of 
neighborhoods and how to efficiently explore these neighborhoods will also be considered. Finally, on top of the local 
search 
there several metaheuristics that can be studied. \medskip \\   
The goal will then be to extend the range of problems that can be solved by looking at more general, non linear 
constraints such as those considered in CP. \medskip \\ 
The performance of the solver will be tested with the instances from the MIPLIB, on timetabling problems and eventually 
on some competitions like the International Nurse Rostering Competition (NERC) and the MiniZinc challenge. 















\end{document}
