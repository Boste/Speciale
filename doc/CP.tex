Constraint programming involves defining variables and constraints like ILP, but often a wide range of constraints can 
be used. Constraint programming uses either a programming language or a framework that have procedures implemented to 
solve the problem according to the constraints posted. The language or framework may provide global constraints that 
can be used to formulate the problem. An example of a global constraint is the {\sffamily{alldifferent(\textbf{x})}} 
constraint that specifies the variables \textbf{x} 
must have pairwise distinct values.  \\ 
Two important aspects of solving a CSP are inference and search. \emph{Inference} is adding constraints to the CSP that 
does not eliminate any feasible solution but might make it easier to solve the CSP \cite[p.301]{CPbog}. 
Local constraint propagation is an example of inference when dealing with variables with finite domain and can be used 
to eliminate large subspaces of the search space $S$. Propagation can be restricting domains of variables, called 
filtering, or combinations of values to variables, based on the constraint doing propagation \cite[p. 169]{CPbog}. 
Propagation can be done when a constraint is created by eliminating values from the domain of variables. I.e. by 
doing propagation on the constraint $x_1 + x_2 \leq 2$ where $x_1,x_2 \in \mathbf{Z}^+$ we can reduce the domain of the 
variables to $\{0,1,2\}$. If we set $x_1 =1$ we can again do propagation and reduce the domain of $x_2$ to $\{0,1\}$. 
\\ 
Search strategies explores possible assignments of variables and an exhaustive search would be a combination of all 
possible assignments of values to the variables. When combining propagation and search strategies the search space can 
be examined exhaustively and large subspaces can be pruned by propagation. \\ 
