\documentclass[a4paper,10pt]{article}
\usepackage[utf8]{inputenc}

%opening
\title{}
\author{}

\begin{document}


\section{Literature}
Different articles/books that I have read are listed here with summary/comments. 
\subsection{Rossi chp 4.6.1}
Finding the order that variables should be assigned values is very difficult. Heuristics are used for this. Most basic 
heuristic is to take the variable that has the smallest domain available. Another approach is the first fail, take the 
path where you are most likely to fail. More complex heuristics are domain divided with degree or weighted degree. \\
Like the approach with the smallest domain one can choose the most constraint variable. One could translate the 
problem into a graph structure and use a structure-guide heuristic, fx the CSP (constraint satisfactory problem?) can 
be modeled as to variables are neighbors if the are in the same constraint. If there is no cycles it is more efficient 
than domain degree heuristics. The last approach is to divide the graphs into subgraph s with a minimal number of cuts. 
\subsection{Estellon2009}
Suggested approach to engineering local search is in three parts. Search strategy, moves and algorithmics \& 
implementation. One can increase or decrease the search space by mooving constraints to obj func or relaxing 
constraints. The search graph is connected to the search space and an algorithm. The graph can be expanded by 
introducing different moves. This should not be controlled sololy by meta heuristics but through redefinition of search 
space and different moves. Obvious thins as optimizing code by cache allocation speed up the search. \\ 
Verify and test your code. Implement debug mode, use assert and calculate function with naive method and compare. \\
After these steps use statistic on moves to improve code (80-20) rule. 
\subsection{Benoist2009}
Basically using the thing proposed in Estellon2009 and put some number for their specific problem that worked out well. 

\subsection{oscaR}
A backend constraint based local solver CBLS for miniZinc based in oscaR. Good for inspiration when describing local 
search. Creating a CBLS is done in 4 steps. Simplifying the model (reducing domain of variables), Classify constraints 
(soft, implicit or one-way), The actual model and search procedure. fzn-oscar-cbls uses basic propagation but want to 
extend it so it uses oscaR's propagation until fix point. One-way constraints are made greedily in 5 steps and ordered 
in a digraph where one-way constraints are nodes and there is an arc from a to b if b is dependent on a. It may not 
contain cycles so the cycles are broken by making a constraint non-one-way. The one with the lowest domain is chosen 
ties a broken by taking the constraint with largest number of input varialbes defined by other invariants. Finding the 
smallest number of arc is the minimum feedback arc set problem and is NP-hard. \\ 
Neighbourhood depends on the type of the constraint because the moves are different. \\ 
The search procedure is three steps. Greedy initial, tabu search to find a solution and another tabu for improving 
solution. \\
The results ar epromising though not as good as other backend there was some hard instances it performed well. Good 
analysis of which instance types was hard to solve. 
\end{document}
