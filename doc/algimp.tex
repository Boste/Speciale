When a model better suited for local search has been made, the remaining time is used to do local search. The 
algorithms check if the timelimit is reach before each iteration of a search procedure. The best solution found while 
searching is saved in a \class{State} class such that the search can continue but always report the best solution when 
the time limit is reached. \\ 
Three algorithms has been made from combining \class{Neighborhood} classes and search procedures that will be used to 
test efficiency of the framework. The first algorithm uses two \class{TabuSearch} with different 
\class{Neighborhood} classes. When the solution is infeasible \class{TabuSearch} is combined with \class{ConflictOnlyNE} 
to only look at variables that can reduce the number of violations. When the current solution is feasible 
\class{TabuSearch} with a \class{RestrictedFlipNE} class is used. If the number of independent variables are less than 
or equal to 5000 it uses a \class{FlipNeighborhood} class instead. The reason for choosing a subset of the neighborhood 
to examine is to increase the number of iterations made in case the neighborhood is large. \\
\IncMargin{1em}
\begin{algorithm}[H]

%\SetKwFunction{relation}{relation}\SetKwFunction{coeff}{coefficient}
  \algdata
%\Input{\int iteration, int[] best, int[] current, int[] tabulist }
%\Output{\bool improvement}
%\BlankLine
\class{ConflictOnlyNE} CON \;
\class{RestrictedFlipNE} RFN \;
\class{TabuSearch} TSCON(NE) \;
\class{TabuSearch} TSRFN(NE2) \;
\int iteration = 0 \;
\int[] best = getSolution() \;
\int[] current = getSolution() \;
\int[] tabulist(neighborhood.getSize(), -neighborhood.getSize()) \;
\While{\upshape within time limit}{
  \eIf{\upshape Current solution is feasible}{
    TSCON.start(iteration ,current, best, tabulist) \;
    iterations++ \;     
    \If{\upshape getSolution() is better than bestSolution} {
      bestSolution = getSolution() \;
    }
  }{
    TSRFN.start(iteration, current, best, tabulist) \;
    iterations++ \;     
    \If{\upshape getSolution() is better than bestSolution} {
      bestSolution = getSolution() \;
    }
  }
}
\caption{Local Search - Test Algorithm 1} \label{algo_LS1} 
 %\caption{Test if a constraint $c$ can define a variable $x$ } \label{algo_checkoneway}
\end{algorithm} \noindent
\DecMargin{1em} \\
The second algorithm for testing is iterated local search using first improvement and random walk \cite[p. 
141-142]{lsbog} with a single flip neighborhood. They idea is to find a local optima fast, and use randomness to escape 
the optima. The random walk chooses a random neighborhood operation and commits that if it is legal. This is repeated 
a number of time depending on the size of the neighborhood. This is done to try to explore many different subspaces 
of the search space $S$. \\ 
\IncMargin{1em}
\begin{algorithm}[H]

%\SetKwFunction{relation}{relation}\SetKwFunction{coeff}{coefficient}
  \algdata
%\Input{\int iteration, int[] best, int[] current, int[] tabulist }
%\Output{\bool improvement}
%\BlankLine

\class{FlipNeighborhood} FN \;
\int randomMoves = min(FN.getSize() / 50, 10) \;
\class{FirstImprovement} FI(FN) \;
\class{RandomWalk} RW(FN, randomMoves) \;
\While{\upshape within time limit}{
\bool improvement = true  \;
  \While{improvement \textbf{AND} \upshape within time limit}{
    improvement = FI.start() \;
    \If{\upshape getSolution() is better than bestSolution} {
      bestSolution = getSolution() \;
    }
  }
    RW.start() \;
    \If{\upshape getSolution() is better than bestSolution} {
      bestSolution = getSolution() \;
    }
  
}
\caption{Local Search - Test Algorithm 2} \label{algo_LS2} 
 %\caption{Test if a constraint $c$ can define a variable $x$ } \label{algo_checkoneway}
\end{algorithm} \noindent
\DecMargin{1em} \\
The last algorithm that will be tested uses a minimum conflict heuristic with a single flip neighborhood when the 
current solutions is infeasible. When the current solution is feasible a tabu search with a restricted single flip 
neighborhood is used. \\ 

\IncMargin{1em}
\begin{algorithm}[H]

%\SetKwFunction{relation}{relation}\SetKwFunction{coeff}{coefficient}
  \algdata
%\Input{\int iteration, int[] best, int[] current, int[] tabulist }
%\Output{\bool improvement}
%\BlankLine
\class{RandomConflictConNE} RCC \;
\class{RestrictedFlipNE} RFN \;
\class{BestImprovement} BIRCC(RCC) \;
\class{TabuSearch} TSRFN(RFN) \;
\int iteration = 0 \;
\int[] best = getSolution() \;
\int[] current = getSolution() \;
\int[] tabulist(neighborhood.getSize(), -neighborhood.getSize()) \;
\While{\upshape within time limit}{
  \eIf{\upshape Current solution is feasible}{
    BIRCC.start() \;
    iterations++ \;     
    \If{\upshape getSolution() is better than bestSolution} {
      bestSolution = getSolution() \;
    }
  }{
    TSRFN.start(iteration, current, best, tabulist) \;
    iterations++ \;     
    \If{\upshape getSolution() is better than bestSolution} {	
      bestSolution = getSolution() \;
    }
  }
}
\caption{Local Search - Test Algorithm 3} \label{algo_LS3} 
 %\caption{Test if a constraint $c$ can define a variable $x$ } \label{algo_checkoneway}
\end{algorithm} \noindent
\DecMargin{1em} \\
Several other algorithms can be made from the \class{Neighborhood} and search procedure classes. Though they can be 
combined in many ways they are not as easily combined as wanted. The \class{Neighborhood} classes can be a mix of 
heuristics and a neighborhood which is not ideal. %The next subsection will discuss this in more detail. 



