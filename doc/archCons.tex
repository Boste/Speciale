Constraints have some properties in common which is implemented in the parent class. All constraints have some 
measurement of violation. The violation can either be a zero-one measurement or it can be a measurement of how far 
from satisfied it is \boste{or how ``oversatisfied'' it is}. All constraints implement must overload the methods 
``setDeltaViolation'' and ``updateViolation'' from the parent class. These methods are only used during local search but 
are need in order to evaluate a move. \boste{Define Move before this?}. The method ``setDeltaViolation'' calculates how 
much a constraint would change in violation if the move proposed is made. The method ``updateViolation`` is used to 
update the current violation of a constraint. \\
A user can give the constraints a priority when posting the model. This priority is used as a measurement to which of 
the constraints should be satisfied first. \\ 
\boste{This section feels rather redundant since it could be done by invariants as well}