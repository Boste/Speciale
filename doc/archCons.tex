Constraints have some properties in common which is implemented in the parent class. All constraints have some 
measurement of violation. The violation can either be a zero-one measurement or it can be a measurement of how far 
from satisfied it is \boste{or how ``oversatisfied'' it is}. All constraints implement must overload the methods 
``setDeltaViolation'' and ``updateViolation'' from the parent class. These methods are only used during local search but 
are need in order to evaluate a move. \boste{Define Move before this?}. The method ``setDeltaViolation'' calculates how 
much a constraint would change in violation if the move proposed is made. The method ``updateViolation`` is used to 
update the current violation of a constraint. \\
A user can give the constraints a priority when posting the model. This priority is used as a measurement to which of 
the constraints should be satisfied first. \\ 
\boste{This section feels rather redundant since it could be done by invariants as well} \\ \\ 
Constraints are all derived from the same class that force some method and parameters to be implemented. All 
constraints needs a priority according to how important the constraint is. The priority do not need to be different for 
the constriants but it will help the local search to differentaite between solution. 
\boste{which? Maybe a UML like box that shows the method and most important parameters} \\ 
A constraint is posted in the constraint programming environment and later handled by local search environment. The 
constraints are treated differently in the environments and need different parameters and methods for that. For the 
CP environment few special parameters are need such as the integer consistency level (ICL) \boste{What more?}. The LS 
environment handles constraints through invariants hence a constraint needs a method for creating the invariant(s) 
needed in LS for the specific type of constraint. \\ 
\boste{Example} \\
The Linear constraint has by definition variables, coefficients, a relation and a right hand side. When posting the 
constraint in Gecodes environment an ICL argument can be useful for guiding Gecode. In the LS environment the 
constraint is handled by creating two invariants, one for the value of the left hand side and one to determine if the 
constraint is violated. Implementation of invariants is described in next subsection and how the invariants are created 
is described in section \ref{sec_ls}. 