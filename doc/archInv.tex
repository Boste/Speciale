Invariants are all derived from one common class just like constraints. Invariants are only introduced after an initial 
solution to the problem is found and before the local search has begun. Invariants can represent either variables or 
auxiliary variables and are defined by oneway constraints. All Invariants have the methods \method{addChange} and 
\method{calculateDelta} that are used during local search. The \method{addChange} is used to tell an invariant that a 
variable it depends on has been changed. The \method{calculateDelta} is used to update the invariant according to the 
changes recieved. \\ 
Each type of invariant must implement its own method since the methods can be different for each type of invariant. \\ 
Some of the classes created in the Local Search Engine uses invariants but do not differentiate between them. If 
invariants did not have a common parent class then each invariant type would need its own data structure for storage. 
Another benefit is the search procedures do not have to exam which invariants the model consist of since they all have 
the same methods. It also makes it easier to add new invariants since all the functionality is implemented by the new 
invariant and nothing has to be changed in the Local Search Engine. \\ 
 

