This section gives an overview of the key components of this solver. Figure \ref{fig_architec} gives an overview 
of the most basic classes and the classes they store pointers to i.e. have access to. 
\begin{figure}[!b]
\includegraphics[width=\linewidth]{architectureTest}\caption{Overview of what the key classes of the framework have 
access to.} 
\label{fig_architec}
\end{figure} \noindent
The two engines of the framework are the \gecodesol and \lssol which find the initial solution and optimize the 
solution respectively. \gecodesol is used for preprocessing and finding an initial solution, if possible with in the 
limits given. \gecodesol will be elaborated on in section \ref{sec_gecode}. \\
\lssol is responsible for the optimization part of the solver with the use of local search and 
metaheuristics. How the optimization is perfomed is described in section \ref{sec_local}. \lssol 
transforms the model to a model better suited for local search before the local search is started. How this is done and 
why will be discussed in section \ref{sec_ls}. \\ 
The \class{Storage} class contains pointers to components of a CBLS model, variables, constraints, and invariants. 
\lssol can add new objects such as invariants to \class{Storage}. \\ 
\class{Constraint} and \class{Invariant} are super classes to all constraints and invariants respectively. They are 
described in subsection \ref{sub_cons} and \ref{sub_inv}. They contain abstract methods which the subclasses must 
define. \\ 
The main part of the solver is the \class{General Solver} class that contains the engines used for solving.
The \class{General Solver} class contains the methods that are called by the user, such as creation of variables and 
constraints, finding initial solution and optimizing the solution and is described in subsection \ref{sub_gen}. 

