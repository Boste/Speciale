This section gives an overview of the key components of the solver. The figure \ref{fig_architec} gives an overview 
of the most basic classes and the classes they have ponters to (access to). \\
\begin{figure}[!b]
\includegraphics[width=\linewidth]{architectureTest}\caption{Overview of what the classes contains pointers to} 
\label{fig_architec}
\end{figure}
The two engines for solving are the \gecodesol and \lssol that find the initial solution and optimize the solution 
respectively. \gecodesol is used for preprocessing and finding an initial solution, if possible with the limits given. 
\boste{Either just timelimit or could be made visible to the user by an option class with node, fail, and time limit.}  
\gecodesol will be elaborated further in section \ref{sec_gecode}. \\
\lssol is responsible for the optmization part of the solver with the use of local search and metaheuristics. \lssol 
transform the model to a model better suited for local search before the local search can start. How this is done and 
why will be discussed in section \ref{sec_ls}. \\ 
The \class{Model} class contains pointers to components of a CBLS model, variable, constraints, and invariants. The 
engines naturally has access to these objects and \lssol can add new objects such as invariants. \\ 
\class{Constraint} and \class{Invariant} are superclasses to all constraint and invariant respectively and are 
described in subsection \ref{sub_cons} and \ref{sub_inv}. They contain abstract methods that the subclasses must 
specify. \\ 
The main part of the solver is the \class{General Solver} class that contains the engines used for solving.
The \class{General Solver} class contains the methods public to the user, such as creating variables and constraints, 
finding initial solution and optimizing the solution. \\ 
The \class{Variable} class contains both the variable used by Gecode but is also used for local search. 
