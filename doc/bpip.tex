Binary- and integer linear programming can be used to model a wide range of problems by posting linear constraints and 
using and a linear objective function. A linear integer program can be writing on the form: 
\begin{align}
 \text{Minimize }\; &z =  \mathbf{c}^T\mathbf{x} \\ 
 \text{subject to } \; & A\mathbf{x} \leq \mathbf{b} \\ 
 & \mathbf{x} \in \mathbb{Z}^n
\end{align}
Here $A$ is a $n \times m$ matrix of coefficients, $\mathbf{b} \in \mathbb{R}^m$, $z$ is the value of the objective 
function and $\mathbf{c} \in \mathbb{R}^n$. The first line is the objective function and can easily be transformed to a 
maximization problem by multiplying by $-1$. The relation in line 2 can be a mix of $\{\leq,=,\geq\}$ but $\geq$ can 
be transformed to $\leq$ by multiplying both side of the constraint by $-1$.  \\ 
A solution is an assignment of values to all variables $\mathbf{x}$ and a solution is said to be feasible if all 
constraints are satisfied. The set of feasible solutions consist of integer point in a $n$ dimensional space and the 
point that minimize the objective function is said to be the optimal solution. \\
Solving a general integer program or binary program is a NP-hard problem and several techniques are developed for 
solving them. The techniques can be i.e. branch and bound, cutting plane and branch and cut and usually by
relaxing the integer constraint to solve an auxiliary linear problem \cite[p. 30]{ilp}. The linear problem can be 
formulated as the integer linear problem where variables can take real values. The linear formulation can be 
transformed to standard form by introducing auxiliary variables and then solved using the simplex method. The relaxed 
problem can be used as a lower- or upper bound depending on whether the problem is a minimization or maximization 
problem. \boste{Jeg ved ikke om jeg skal uddybe det mere, når jeg kun beskæftiger mig med BP/IP} \\ 
There exist several solver for (integer) linear programming problems such as Cplex, Gurobi, GLPK and Scip. 