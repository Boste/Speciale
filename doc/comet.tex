

\subsection{Comet}
% Figur om opbygning af comet som i bogen
Comet is an object orientated programming language that uses the modeling language of constraint programming and uses a 
general purpose local search solver. Comet is now an abandoned project but the architecture used is still of interest. 
The core of the language is the incremental store that contains various incremental objects fx. incremental variables. 
Invariants, also called one-way constraints, are expressions that are defined by incremental variables and a relation of 
those. An incremental variable $v$ can for instance be expressed as a sum of other variables. The variable $v$ will 
automatically be updated if one of the other variables changes value. The order in which the invariants gets updated can 
be implemented to achieve higher performance. \\ \medskip
One layer above the invariants is the differentiable objects that can use the invariants and the incremental objects. 
Both constraints and objective are implemented as differentiable objects. They are called differentiable 
because it is possible to compute how the change of a variable value will affect the differentiable object's values. 
All 
constraints are implemented using the same interface, that means that all constraint have some methods in common. These 
method is defined as invariants hence they are always updated when a change is made to one of their variables. This is 
especially useful when combining multiple constraints in a constraint system, that also implements the constraint 
interface. The constraints can be combined in a constraint system that then uses the method from the individual 
constraints to calculated its own methods. Just like the \interface{constraint} interface there exist a 
\interface{objective} interface. Both 
will be described further in section \ref{difObj}. \\ \medskip
The next layer is where the user models their problem and use the objects mentioned above. Several search method are 
implemented that can be used. The benefit of this architecture is the user can focus on modeling the problem 
efficiently on a high level and thereby avoid small implementation mistakes. Using constraint programming inspired 
structure gives the benefit of brief but very descriptive code.

\subsubsection{Invariants}



\subsubsection{Differentiable Objects} \label{difObj}
Comets core modeling object is the differentiable objects that are used to model constraints and objective functions. \\
All constraints in Comet implement the %\il{Constraint} interface (Hvad er et interface?) that makes it 
possible to combine the constraints and reuse them easily. A constraint has at least an array $a$ of 
variables as argument, that is the variables associated with the constraint. The interface specifies some basic 
methods that all constraints must implement such as \method{violations()} that returns the number 
of violations for that constraint and \method{isTrue()} that returns whether the constraint is satisfied. How the number 
of violations is calculated depends on the implementation of the constraint. An example could be the 
\method{alldifferent(a)} constraint, that states that all variables in the array $a$ should have distinct values. The 
method \method{violations()} then returns the number of variables that do not have a unique value. \medskip \\ 
% variables, value or decomposition based violations. Skal uddybes
One of the benefits that all constraints implements the same interface is the ease of combining these constraints. 
%%% Det her skal enten omskrives ingen logical combinators eller udlades.
This can be done by using other differentiable objects such as logical combinators. These objects has a specific 
definition of the basic methods from the \interface{Constraint} interface that means they do not rely on the semantics 
of the constraints to combine them. 
Another way to combine the constraints is to use a constraint system. Constraint systems are containers objects that 
contains any number of constraints. These constraints are linked by logical combinators, namely conjunction. Comet can 
use several constraints system simultaneously and is very useful for local search where one system can be used for hard 
constraints another for soft constraints. Other constraint operators in comet are cardinality constraints, weighted 
constraints and satisfactions constraints. \medskip \\
Objectives are another type of differentiable objects and the structure is very close to the structure of constraints. 
Objectives implements an interface \interface{Objective} that have some of the same methods as \interface{Constraint} 
but instead of having \method{isTrue()} and \method{violations()} they have \method{value()} and \method{evaluation()}. 
The method \method{value} returns the value of the objective but it can be more convenient to look at the method 
\method{evaluation()} to guide the search. The method \method{evaluation()} should indicate how close the current 
assignment of variables is to 
improve the value of the objective or be used to compare two assignments of variables. How the evaluation should do 
that depends on the nature of the problem and could for instance use a function that would decrease as the assignment 
of variables gets close to improve the \method{value()}.
