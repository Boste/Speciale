% \in C$ is a pair $\langle R_{X(c_j)}, X(c_j) \rangle $ where $ R_{X(c_j)}$  also 
%called the relation on $c_j$. \\ 
The values of variables will be restricted by a set of $m$ constraints $C= \{ c_1,c_2, \dots , c_m \} $ and let $J = 
\{1,2,\dots,j,\dots , m\}$. The set of 
variables to which the constraint $c_j$ applies is called its \emph{scope} and is denoted 
%$X(c_j) = \{x_{j,{i_1}}, x_{j,i_2} , \dots , x_{j,\alpha_j}\}$. 
$X(c_j) = \{x_{1j}, x_{2j} , \dots , x_{\alpha_jj}\}$. The variable $x_ij$ is the i'th variable in constriant $c_j$ 
and corresponds to a variable $x_k\in X$.  
The size of a scope $|X(c_j)|$ is called the \emph{arity} $\alpha_j$. The constraint $c_j$ can be expressed as a 
subset of the Cartesian product of the domains of the variables in the scope $X(c_j)$, i.e, $c_j \subseteq 
D(x_{j,i_1}) \times D(x_{j,i_2}) \times \dots \times D(x_{j,i_{\alpha_j}})$. \\ 
If all variables of a constraint $c_j$ has a finite domain then the constraint can be written in extensional form. The 
\emph{extensional form} of $c_j$ is a subset of $\mathbb{Z}^{|X(c_j)|}$ of all combinations of tuples that satisfies 
$c_j$. \boste{Same som cartesian product?} \\ 
We call a constraint $c_j$ a \emph{functional constraint} if given an assignment of values to all variables except 
$x_i$ in $c_j$, then at most one value of $x_i$ satisfy $c_j$ for all $x_i \in X(c_j)$. In other words the value of a 
variable in a functional constraint can be determined from the values of the other variables in the functional 
constraint. \boste{Equational constraint?}

\iffalse
\subsubsection{Considered Constriants \boste{Not sure about the title and not finished yet. Maybe move to ``Structure 
of this CBLS''}}
While Constraint Programming often offers a wide selection of constraints to use, this work focuses mostly on the 
constraint \Linear that is defined by a left hand side, a relation $R$ and a 
right hand side, which is a bound $b$. The left hand side is a linear function of decision variables 
multiplied by coefficients. The relation $R$ between left hand side and right hand side is restricted to be one of 
the six $R \in \{<,\leq,>,\geq,=,\neq\}$ \\
%less ($<$), less or equal ($\leq$), greater($>$), greater or equal($\geq$), equal ($=$), and disequal 
%($\neq$). \boste{Ikke særlig pænt, men ved ikke hvordan jeg 
%skal beskrive det ellers (Gecode har seks, MIP og IP har kun 3)}\\ 
A linear constraint $c$ can be described as: 
\begin{equation}
 \sum\limits_{x_j \in X(c)} a(x_j) \cdot x_j \;\; R \;\; b(c) \qquad R \in \{<,\leq,>,\geq,=,\neq\}%\lesseqgtr B_c
\end{equation}
The coefficients $A(c)$ are the coefficients of the variables in the scope of $c$. The decision variables $X(c)$ are 
the variables that $c$ applies to. The bound $B_c$ is the bound for the left hand side in constraint $c$. \\
\boste{Follwing should be a note about CSP is a superset of MIP,IP,BP}
MIP, IP and BP are restricted to use the linear constraint and the model can be written as: 
\begin{align}
 \text{Minimize } & \mathbf{c}^T\mathbf{x} \nonumber \\ 
 \text{Subject to } &\mathbf{Ax} \leq \mathbf{b} \nonumber\\
 &\mathbf{x} \in \mathbf{D}
\end{align}
\fi




%In order to define mixed integer programming and integer programming we need to define a function $f(X) = 
%\sum_{x_i \in X} 
%c_i x_i$ where $c_i \in E 


%We can reduce the CSP to 3-SAT by restricting the domain of the variables to zero and one and restrict the scope of 
%all constraints to exactly three. By that we can conclude that some CSP are NP-complete and not easy to solve.  