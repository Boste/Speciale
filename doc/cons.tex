% \in C$ is a pair $\langle R_{X(c_j)}, X(c_j) \rangle $ where $ R_{X(c_j)}$  also 
%called the relation on $c_j$. \\ 
The operant of values to variables will be restricted by a set of $m$ constraints $C= \{ 
c_1,c_2, \dots , c_m \} $. The set of variables to which the constraint $c_j$  \boste{Drop subscript j?} applies is 
called its \emph{scope} and 
is denoted $X(c_j) = \{x_{j,1}, x_{j,i} , \dots \}$. The size of a scope $|X(c)|$ is called the 
\emph{arity} $\alpha(c)$. The constraint $c_j$ is a subset of the cartesian product of the domains of the variables in 
the scope $X(c_j)$ of $c_j$, ie, $c_j \subseteq D(x_{i_1}) \times D(x_{i_2}) \times \dots \times 
D(x_{i_{\alpha(c_j)}})$. \\
The Constraint Satisfaction Problem (CSP) can then be defined as a triple $\mathbb{P} = \langle X,D,C \rangle$. A 
\emph{solution} to the CSP $\mathbb{P}$ is a vector of n elements $\Tau = (\tau_1,\tau_2,\dots,\tau_n) $ 
where $\tau_i 
\in D_i$. The solution is feasible if $\Tau$ can be projected \boste{definition needed} onto $c_j$ for all $c_j \in C$. 
\boste{Er det helt rigtigt?}\\ 
The questions to a CSP could be to report all feasible solutions $sol(P)$, any feasible solution $\mathbf{\tau}\ in 
sol(\mathbb{P}$ or if there 
exists a solution $\mathbf{\tau}$ or not. \\
The CSP $\mathbb{P}$ can be expanded to a Constraint Satisfaction Optimization Problem (CSOP) $\mathbb{P'}$ 
with an objective function $f(\mathbf{\tau})$ that evaluates the quality of the solution $\mathbf{\tau}$, $\mathbb{P'} 
= \langle 
X,D,C,f(\mathbf{\tau}) \rangle$. The task is then to find a 
solution $\hat{\mathbf{\tau}}$ that gives minimum or maximum value of $f(\hat{\mathbf{\tau}})$ depending on the 
requirements of the 
problem. 
\iffalse
\subsubsection{Considered Constriants \boste{Not sure about the title and not finished yet. Maybe move to ``Structure 
of this CBLS''}}
While Constraint Programming often offers a wide selection of constraints to use, this work focuses mostly on the 
constraint \Linear that is defined by a left hand side, a relation $R$ and a 
right hand side, which is a bound $b$. The left hand side is a linear function of decision variables 
multiplied by coefficients. The relation $R$ between left hand side and right hand side is restricted to be one of 
the six $R \in \{<,\leq,>,\geq,=,\neq\}$ \\
%less ($<$), less or equal ($\leq$), greater($>$), greater or equal($\geq$), equal ($=$), and disequal 
%($\neq$). \boste{Ikke særlig pænt, men ved ikke hvordan jeg 
%skal beskrive det ellers (Gecode har seks, MIP og IP har kun 3)}\\ 
A linear constraint $c$ can be described as: 
\begin{equation}
 \sum\limits_{x_j \in X(c)} a(x_j) \cdot x_j \;\; R \;\; b(c) \qquad R \in \{<,\leq,>,\geq,=,\neq\}%\lesseqgtr B_c
\end{equation}
The coefficients $A(c)$ are the coefficients of the variables in the scope of $c$. The decision variables $X(c)$ are 
the variables that $c$ applies to. The bound $B_c$ is the bound for the left hand side in constraint $c$. \\
\boste{Follwing should be a note about CSP is a superset of MIP,IP,BP}
MIP, IP and BP are restricted to use the linear constraint and the model can be written as: 
\begin{align}
 \text{Minimize } & \mathbf{c}^T\mathbf{x} \nonumber \\ 
 \text{Subject to } &\mathbf{Ax} \leq \mathbf{b} \nonumber\\
 &\mathbf{x} \in \mathbf{D}
\end{align}
\fi




%In order to define mixed integer programming and integer programming we need to define a function $f(X) = 
%\sum_{x_i \in X} 
%c_i x_i$ where $c_i \in E 


%We can reduce the CSP to 3-SAT by restricting the domain of the variables to zero and one and restrict the scope of 
%all constraints to exactly three. By that we can conclude that some CSP are NP-complete and not easy to solve.  