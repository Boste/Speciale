A \emph{Constraint Satisfaction Problem} (CSP) is defined as a triple $\mathbb{P} = \langle X,D,C \rangle$. A 
\emph{candidate solution} to a CSP $\mathbb{P}$ is a vector of $n$ elements \\
%$\tau = (\tau_1,\tau_2,\dots,\tau_n) $ where $\tau_i \in D_i$. 
$\tau = (V(x_1), V(x_2), \dots , V(x_n))$ from the set of all candidate solutions $S$ called the \emph{search space}.  
% D_1 \times D_2 \times %\dots , D_n) $ where $\tau_i \in D_i$. 
Given a sequence $X' \subseteq X$ of variables $\tau[X']$ is called a restriction on $\tau$, ordered according to 
$X$. The constraint $c_j$ is satisfied by $\tau$ if the restriction $\tau[X(c_j)]$ matches a tuple of the constraint 
$c_j$ in extensional form. If each constraint $c_j \in C$ is satisfied then the solution $\tau$ is a \emph{feasible 
solution} to the CSP $\mathbb{P}$. \\

