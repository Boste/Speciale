A \emph{Constraint Satisfaction Problem} (CSP) is defined as a triple $\mathbb{P} = \langle X,D,C \rangle$. A 
\emph{candidate solution} to a CSP $\mathbb{P}$ is a vector of n elements 
%$\tau = (\tau_1,\tau_2,\dots,\tau_n) $ where $\tau_i \in D_i$. 
$\tau = (V(x_1), V(x_2), \dots , V(x_n))$.  % D_1 \times D_2 \times \dots , D_n) $ where $\tau_i \in 
%D_i$. 
Given a sequence $X' \subseteq X$ of variables $\tau[X']$ is called a restriction on $\tau$, ordered according to 
$X$. If the restriction $\tau[X(c_j)]$ matches a tuble of the constraint $c_j$ in extensional form the solution $\tau$ 
satisfies constriant $c_j$. If each constraint $c_j \in C$ is satisfied then the solution $\tau$ is a \emph{feasible 
solution} to the CSP $\mathbb{P}$. \\
\boste{Har tænkt på et flytte nedenstående til en ny section (Solutions) og introducere search space} \\ 
For a CSP the questions of interest could be to report all feasible solutions $sol(P)$, any feasible solution 
$\mathbf{\tau}\in sol(\mathbb{P})$ or if there exists a solution $\mathbf{\tau}$ or not. \\
The CSP $\mathbb{P}$ can be expanded to a \emph{Constraint Optimization Problem} (COP) $\mathbb{P'}$ 
with an objective function $f(\mathbf{\tau})$ that evaluates the quality of the solution $\mathbf{\tau}$, $\mathbb{P'} 
= \langle X,D,C,f(\mathbf{\tau}) \rangle$. The task is then to find a feasible solution $\hat{\mathbf{\tau}}$ that 
gives minimum or maximum value of $f(\hat{\mathbf{\tau}})$ depending on the requirements of the problem. 
