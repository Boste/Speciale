Gecode (GEneric COnstraint Development Environment) is a constraint programming solver implemented in C++ and 
offer a wide range of modeling features. Gecode offers more than 70 constraints from the ``Global Constraint Catalog'' 
\cite{url_globalCons} that can be applied to boolean, integer, set and float variables. \boste{Muligvis Gecode 
architecture billede} \\ 
A model created for Gecode is created by inheriting the \class{Space} class. \class{Space} is a class in Gecode that 
a user can create the model in. To create variables or post constraints the user need to specify the \class{Space} they 
should be created in. When variables are created in a \class{Space}, views are created and associated with the 
variables. Views are not used in modeling but are used to know when propagation should be made on a constraint. \\ 
When posting constraints in a \class{Space}, Gecode creates propagators and these propagators can subscribe to the views 
of the variables in the constraint. When variables changes domain the corresponding view tell its subscribes that the 
variables domain has changed. For some constraint the user has the option to choose the propagator based on a 
consistency level. The cost of different consistency level varies from linear $O(n)$ in the number of variables $n$ to 
exponential $O(D(x)^n$. \cite[p.57]{MPG:M}. \\
To solve a problem Gecode needs guidance when searching and that is done by a \method{branch} method. Once a problem 
has been formulated, the user must define on which variables and how branching is done. Just like variables and 
constraints are 
posted in a \class{Space} the branch order is also posted on the \class{Space}. The choices in branching for a set of 
variables are 
which of those variables to branch on first and what values to branch on. One can post several branch methods and they 
are treated in the order they are posted. Once all variables have been branched in one \method{branch} method it 
continues with the next \method{branch} or restarts.  If no branch strategy is chosen for a variable then branching is 
not done on that variable. \\ 
To start the search a Gecode search engine must be chosen and Gecode offers two, a depth first search engine and a 
branch and bound engine. Search engines have an option class in which several options can be set \cite[p.157]{MPG:M}. 
\\ 
When searching for a solution in a \class{Space}, the search can be illustrated as a binary tree where the edges are 
branch choices for a variable and the vertices are the \class{Space} created because of those choices. If it reaches a 
point 
where no feasible solution is possible it stops branching from that vertex and the \class{Space} is said to be failed. 
Gecode creates clones of \class{Space}s while searching for a solution that are used for backtracking. When Gecode 
reaches a failed \class{Space}, instead of starting from scratch and recompute all way to down to the previous vertex, 
it 
uses the closest clone to backtrack to that \class{Space}. \\