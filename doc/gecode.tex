Gecode (generic constraint development environment) is a constraint programming solver implemented in C++ and 
offer a wide range of modeling features. Gecode offers more than 70 constraints from the ``Global Constraint Catalog'' 
\cite{url_globalCons} that can be applied to boolean, integer, set and float variables. \boste{Muligvis Gecode 
architecture billede} \\ 
A model created for Gecode is created by inheriting the space class. Space is is a basic layer in Gecode that a user 
can build the model on. To Create variables or post constraints the user need to specify the space they should be 
created in. When variables are created in a space, views are created and associated with the variables. 
Views are not used in modeling but are used to know when propagation should be made on a constraint.  When posting 
constraints in a space, Gecode creates propagators and these propagators can subscribe to the views of the variables 
in the constraint. When variables changes domain the corresponding view tell its subscribes that the variables domain 
has changed. For some constraint the user has the option to choose the propagator based on a consistency level. 
The cost of different consistency level varies from linear in the number of variables to exponential \cite[p.57]{MPG:M}. 
\\
To solve a problem Gecode needs guidance when searching and that is done by a branch function. Once a problem has been 
formulated, the user must define on which variables and how branching is done. Just like variables and constraints are 
posted in a space the branch order is also posted on the space. The choices in branching for a set of variables are 
which of those variables to branch on first and what values to branch on. One can post several branch methods and they 
are treated in the order they are posted. Once all variables have been branched in one branch function it continues 
with the  If no branch strategy is chosen for a variable then branching is not done on that variable. \\ 
To start the search a search engine must be chosen and Gecode offers two, a depth first search engine and a branch and 
bound engine. Search engines have an option class in which several options can be set \cite[p.157]{MPG:M}. \\ 
When searching for a solution in a space, the search can be illustrated as a binary tree where the edges are 
branch choices for a variable and the vertices are the space created because of those choices. If it reaches a point 
where no solution is possible it stop branches from that vertex and the space is said to be failed. While searching for 
solution sometimes, based on a search parameter from the option given, Gecode clones the spaces. When Gecode 
reaches a failed space, instead of starting from scratch and recompute all way to down to the previous vertex, it 
uses the closest clone to backtrack to that space. \\