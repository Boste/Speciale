Once domain reduction preprocessing has been done a Gecode DFS search engine is started. The stop criteria for Gecodes 
search can be specified by an option class. A Gecode search engine takes a space and search option 
as arguments and the search option contains a stop object. The stop object can either be timestop, nodestop or 
failstop. Each time Gecode branches on a variable two new nodes are created and nodestop set a upper bound on the 
number of nodes to explorer. If Gecode reaches a node that gives an infeasible assignment to a variable then that space 
is failed and failstop sets an upper bound on the number of failed spaces that can occur before stopping. Timestop 
stops the search if the time limit is reach. \\ 
Instead of having only one of these stop object a multistop object has been created that combines all three stop 
objects if the user wants to having multiple stop criteria. \\ 
Combinatorial problems can be formulated with Gecode and these problems can be very difficult to solve. In these cases 
Gecode keeps searching for a solution until it finds one (or runs out of available memory). Instead, the search can be 
stopped using stop object and the constraints can be relaxed such that Gecode can find a initial solution to the 
instance. If one of the stop criteria is reached the \method{relax} is called and some of the constraint are relaxed. To 
relax some constraints a new Gecode Engine is created and all variables and constraints, except those relaxed, 
are created and added to the new space. In order to choose which constraint that should be removed the priority of the 
constraints, given by the user, is used. Different techniques can be applied for choosing which constraints with the 
same priority to remove. The one chosen here is a stepwise backward algorithm, that is a greedy approach. \boste{Article 
Marco talked about propose some method, but says a greedy approach is good as well? }. The number of constraints that 
are relaxed/removed is based on the number of constraints $|C|$ and the number of restarts made $r$. The number of 
relaxed constraints is given by equation (\ref{eq_relaxed}). 
\begin{equation}
 Relaxed = \ceil[\bigg]{\frac{2^{r}}{100}\cdot |C|}
 \label{eq_relaxed}
\end{equation} 
After six relaxations 64 \% of the constraints have been relaxed and if no solution can be found within the search 
limits, the search is stopped and finding a solution to the instance has failed.  \\ 
The constraints are chosen by their priority and ties are broken at random. I.e. a model with 100 constraint of 
priority 1, 40 of priority 2, and 15 of priority 3 and there are 20 constraints that should be relaxed. The 15 
constraints with priority 3 and 5 of the 40 constraints with priority 2 are chosen at random would not be posted. The 
constraints that are not posted in the Gecode space are still applied when doing local search, hence the initial 
solution might start with some violations.  