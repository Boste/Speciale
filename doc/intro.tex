The field of optimization can be split into several subfields depending on the nature of the decision variables 
(continuous, discrete) and the structure of the problem (linear, non linear, combinatorial, convex, non 
convex). The main focus of this thesis is discrete optimization both linear and non linear. 
Several solvers are available for discrete optimization. The mixed integer linear programming (MILP) approach has 
solvers such as \emph{GLPK}, \emph{Gurobi} and \emph{CPLEX}. There are also constraint programming (CP) solvers, such 
as \emph{Gecode}. All these solvers solve the problem exactly, but for some problems thit is not always possible due to 
the computational cost. Another approach is to use local search and find a good solution fast by making a trade off 
between speed and solution quality. \medskip \\ 
There exists a vast literature about how to make good local search solvers for specific problems.
However, only a few attempts have been made to use local search for general purpose solvers like 
mathematical programming and constraint programming. \emph{Comet} was a successful CP based solver and that allowed to 
use local search to find a good solution fast but the project is now abandoned.  \\ 
\emph{OscaR} is another CP based solver that uses local search to find a good solution. 
\boste{OscaR}
% Currently LocalSolver is a commercial 
% product that combines a local search solver with a problem modeling language. It can be used to solve mathematical, 
% constraint, continuous, and non linear programming. \medskip \\
\boste{Overall made}
In this project, a general heuristic solver based on local search has been developed. It uses Gecode to find 
an initial solution and uses local search trying to improve the solution. It can solve problems formulated as binary 
programming problems but can be extended to solve a wider range of problems. Ideas for structuring the framework are 
drawn from Comet, OscaR, and Gecode. \medskip \\
\boste{More specific}
Beside the basic components of local search several elements have been studied to see their effect on the solution.
 One of these elements is preprocessing, with the help from Gecode, that can reduce the size of the search space before 
the local search is started. Other elements that have be studied are invariants and a directed acyclic graphs to 
represent efficiently the dependencies between the variables and invariants. The choice of neighborhoods and how to 
efficiently explore these neighborhoods have been considered. The quality of a solution is evaluated by a vector in 
lexicographic order instead of a single value. The lexicographic order is from a priority given to the constraints that 
will affect the local search. Finally, on top of the local search different combinations of neighborhoods, procedures, 
and metaheuristics has been tested. \medskip \\   
The framework has a solid base from which it can be extended to solve a wide range of problems by implementing 
different constraints and new neighborhoods. \medskip \\
The performance of the solver has been tested with the instances from the MIPLIB 2010 and compared to Gurobi.  


