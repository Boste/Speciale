\emph{Invariants} are variables, whose value are functionally defined by other variables. Invariants are introduced by 
the solver and neither the variable defined nor the variables defining the variable need not to be decision variables 
in the CSP or CSOP but can be auxiliary variables whose value are of interest. \\ 
\emph{One-way constraints} are constraints that defines the value of an invariant. 
\begin{equation}
 x = f(\mathbf{z}) 
\end{equation}
$f(\mathbf{z})$ is a \oneway with a set of input variables $z$ that functionally defines the \invar $x$. \\ \medskip



%currently in the system but they can be new auxiliary variables whose value are of interest. 
\iffalse
The invariants all have generic methods that needs to be defined for the different types of invariants. One of 
these 
methods is for instance \method{getCurrentValue()} that return the value of the invariant. If a variable $v$ 
changes value then the invariants that are defined by $v$ needs to update their current value. This is done 
through the \method{addChange(arguments)} method, \method{calculateDeltaValue()}, and  \method{updateValue()}. They 
informs the invariant something is changed, calculates the change in value and updates the current value respectively. 
Since invariants can be defined by variables that are invariants themselves this can leads to a series of 
updates. Invariants and constraints can built as a directed graph without cycles in 
order to avoid looking at an invariant multiple times when a change is made to a variable. The construction of that 
graph is described in section \ref{updategraph}. \\ 
An example of an invariant is the \method{Sum} invariant that defines an auxiliary variable. The 
\method{Sum} invariant consist of a coefficient set $C$, variable set $V$ and it can have a constant $b$ added. For 
a \method{Sum} invariant $S$ the value is defined as: \\
\begin{equation}
 S = b+ \sum_{\substack{i \in I_S}}  c_i x_i  
\end{equation}


One-way constraints are constraints that is used to define the value of a single variable. The single variable is made 
as an \class{invariant} and can only change when one of the variables that defines it changes. One-way 
constraints reduces the search space \boste{(which should be discussed before this I guess)} since the variable defined 
by a one-way constraint can only be changed by a move indirectly. 

\fi	

%which value is defined by the sum of the variables and invariant  times their coefficient. 
%Invariants are used to create auxiliary variables based on other variables and/or invariants. 