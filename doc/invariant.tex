Invariants are used to maintain variables or auxiliary variables and they are defined by other variables or invariants. 
The invariants all have methods that needs to be defined for the different types of invariants. These method are 
for instance \textbf{getCurrentValue()} that return the value of the invariant. If a variable $v$ or invariant $i$ 
changes value then the invariants that are defined by $v$ or $i$ needs to update their current value. This is done 
through the \textbf{addChange(arguments)} method, \textbf{calculateDeltaValue()}, and  \textbf{updateValue()}. They 
informs the invariant something is changed, calculates the change in value and updates the current value respectively. 
Since invariants can be defined by other invariants this can leads to a series of updates. In order to avoid looking at 
an invariant multiple times when a change is made the variables, invariants and constraints can built as a 
directed graph without cycles. The construction of that graph is described in section \ref{updategraph}. \\ 
An example of an invariant is the \textbf{Sum} invariant that defines a variable or an auxiliary variable. The 
\textbf{Sum} invariant consist of a coefficient vector $C$ and an invariant vector $Y$ and variable vector $X$. For 
a \textbf{Sum} invariant $S$ the value is defined as: \\
\begin{equation}
 S = \sum_{\substack{x_i \in X_S\\ c_i \in C_S }}  c_i x_i + \sum_{\substack{y_i \in Y_S\\ c_i \in C_S }}  c_i y_i 
\qquad \forall i \in I_S
\end{equation}



%which value is defined by the sum of the variables and invariant  times their coefficient. 
%Invariants are used to create auxiliary variables based on other variables and/or invariants. 