When a local search algorithms makes a move from a current solution to new solution we say it commits to the new 
solution. One of the basic local search algorithms is the \emph{iterative improvement} with different pivot rules. 
iterative improvement explores the neighborhood of the current solution or a subset of the neighborhood and uses the 
delta evaluation function to determine which move to make. \\ 
Two basic pivoting rules for iterative improvement are \emph{Best improvement} and \emph{first 
improvement}. Best improvement examine the neighborhood of the current solution with delta evaluation function and 
chooses the solution that gives the best improvement, if any. First improvement examine the neighborhood and commits to 
the first solution that gives an improvement. iterative improvement is repeated until no improving solution exists. \\ 
Several heuristics can be applied to the algorithms for instance by choosing a subset of the neighborhood to examine at 
random when using best improvement or allowing a number of sidewalks. A sidewalk is going from a solution $\tau$ to a 
neighbor solution $\tau'$ where $\delta(\tau') = 0$. First improvement can be modified to \emph{random improvement} that 
chooses an improving solution with a probability $p$ or looks for the next improving solution with probability $1-p$. 
\\
Another basic local search algorithm is the \emph{random walk}. Random walk commits to one of the neighbor 
solution chosen uniformly random. This might lead to worsening solution and/or infeasible solution and is usually 
combined with other heuristics or local search algorithms. \\
Before local search can be done an initial assignment to the variables is needed and an initialization function is 
used for this also called a construction heuristic. 