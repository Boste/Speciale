There are several important aspects of local search and the most basic will be covered here. \\The evaluation function 
$f(\tau)$ evaluate the quality of the candidate solution. 
The \emph{step function} that defines the candidate solutions close to a given candidate solution $\tau$. By applying 
the step function once to candidate solution $\tau$ a new candidate solution $\tau'$ is obtained. Going from one to 
solution to another with the step function is called a \emph{step}. Several solution might be explored before making a 
step and using a \emph{neighborhood operation} defines by the step function.  Local search uses a neighborhood function 
$N(\tau)$ that for each solution $\tau \in \S$ specify a subset of solutions $N(\tau) = \Tau$. The solutions $N(\tau)$ 
is called the neighborhood of $\tau$ and are the set of solution that can obtained by making one step. The cardinality 
of $N(\tau)$ is called the neighborhood size of $\tau$ and depends on the step function. The neighborhood function $N$ 
is symmetric if, $\tau' \subseteq N(\tau)$ if and only if $\tau \subseteq N(\tau')$ for all pairs of $\tau$ and $\tau'$. 
In a problem with $n$ binary variables the step function could be to change the value of a single variable, a flip.  
This can be done on all variable hence the neighborhood size of a candidate solution would be $n$. \\ 
It might be expensive to use the evaluation function to evaluate each solutions, instead a \emph{delta evaluation 
function} $\delta(\tau)$ can be used. The evaluation function recomputes the quality of a solution even if only a few 
variables changes value. The delta evaluation function only computes how much the evaluation function will change by 
going from one solution $\tau$ to another solution $\tau'$, hence $\delta(\tau) = f(\tau')-f(\tau)$. \\
The search space combined with the neighborhood function can be illustrated as a graph $G = (V,E)$. The set $V$ is a 
set of vertices each representing a candidate solutions of $S$ and $E$ is a set of edges connecting a vertex v, 
representing the solution $\tau$, with the vertices representing $N(\tau)$. The graph $G$ is called the neighborhood 
graph and local search does a walk through the neighborhood graph when searching for a solution. \cite[p. 3-5]{lsbog} 
\\ 
Local search needs a \emph{termination criteria} that determines when the search should stop. Sometimes we know what 
an optimal is, i.e. the SAT problem, once we find a feasible solution we can stop. In other cases an optimal solution 
may not be know, several combinatorial problems are not solved to optimality. The termination function can i.e. be based 
on a time limit, number of steps made, or when a locally optimal solution is found. A locally optimal solution for 
a minimization problem is a solution $\hat{\tau}$, such that for each feasible solution $\tau \in N(\hat{\tau})$ 
$f(\hat{\tau}) \leq f(\tau)$. \\
