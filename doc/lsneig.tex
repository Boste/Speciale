There are several important aspects of local search and the most basic will be covered here. \\The \emph{evaluation 
function}
$f(\tau)$ evaluate the quality of the candidate solution. 
The \emph{neighborhood function} $N(\tau)$ that defines the candidate solutions $s \in S$ close to a given candidate 
solution $\tau$. The set of candidate solutions $s$ is said to be the neighborhood of $\tau$ and can be reach by using 
the neighborhood function once, called a \emph{neighborhood operation}. We call it an iteration each time we move from 
one solution to another and Several solution might be explored before making a neighborhood operation. The cardinality 
of $N(\tau)$ is called the neighborhood size of $\tau$. The neighborhood function is symmetric if, $\tau' \subseteq 
N(\tau)$ if and only if $\tau \subseteq N(\tau')$ for all pairs of $\tau$ and $\tau'$. 
In a problem with $n$ binary variables the neighborhood function could be to change the value of a single variable, 
called a flip. This can be done on all variable hence the neighborhood size of a candidate solution would be $n$. \\ 
It might be expensive to use the evaluation function to evaluate each solutions, instead a \emph{delta evaluation 
function} $\delta(\tau)$ can be used. The evaluation function recomputes the quality of a solution even if only a few 
variables changes value. The delta evaluation function only computes how much the value of the evaluation function will 
change by going from one solution $\tau$ to another solution $\tau'$, hence $\delta(\tau) = f(\tau')-f(\tau)$. \\
The search space combined with the neighborhood function can be illustrated as a graph $G = (V,E)$. The 
set $V$ is a set of vertices each representing a candidate solutions of the search space $S$. The set $E$ is a set of 
edges connecting a vertex v, representing the solution $\tau$, with the vertices representing the solutions 
in $N(\tau)$. The graph $G$ is called the \emph{neighborhood graph} and local search does a walk through the 
neighborhood graph when searching for a solution. \cite[p. 3-5]{lsbog} \\ 
Local search needs a \emph{termination criteria} that determines when the search is stopped. Sometimes we know what 
the optimal solution is, i.e. the SAT problem, once we find a feasible solution we can stop. In other cases an optimal 
solution may not be know, several combinatorial problems are not solved to optimality. The termination function can i.e. 
be based on a time limit, number of steps made, or when a locally optimal solution is found. A locally optimal solution 
for a minimization problem is a solution $\hat{\tau}$, such that for each feasible solution $\tau \in N(\hat{\tau})$ 
$f(\hat{\tau}) \leq f(\tau)$. \\
