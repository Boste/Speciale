The idea is to reduce the number of independent variables hence reducing the size of search space and the neighborhoods 
in local search. This does come with a downside that calculation of a variable changing value might take more time. \\ 
The functional constraint used to create a oneway constraint must be satisfied by the initial solution in order to 
create a oneway constraint. Once a oneway constraint is made it defines a variable that is represented by an invariant. 
The invariants value is always within the domain of the variable which corresponds to the functional constraint always 
being satisfied. \\
Even though all \class{Linear} constraints with an equality relation are functional, only those with unit 
coefficients are chosen to be functional for simplicity. I.e. on the form $c: \sum\limits_{i=1}^{\alpha(c)} a_i 
x_i = b(c), \; |a_i|= 1 \; \; \forall i $. If other coefficients were allowed it could create non integer coefficients 
that does not work with the rest of the framework currently. \\ %\boste{Sikkert et spørgsmål til forsvar} \\ 
For each functional \class{Linear} constraint $c_j$ with unit coefficient, two algorithm steps are used to create 
invariants. The first checks if the constraint $c_j$ can be transformed into a oneway constraint and the other 
transforms $c_j$ into a one-way constraint defining $x_i$. \\ 
\IncMargin{1em}
\begin{algorithm}[H]

%\SetKwFunction{relation}{relation}\SetKwFunction{coeff}{coefficient}
\SetKwFunction{makeOneway}{makeOneway}
\SetKwFunction{defi}{defines}
\algdata
 \Input{\cons $c_j$}
%\Output{Boolean}
\BlankLine
%\If{c \upshape already defines a oneway constraint}{\Return{} \false\; \boste{This constraint could be removed in 
%$O(\alpha(c_j))$}}
% \If{\upshape Number of integer variables not defined $> 1$}{\Return{} \false\; \boste{Needed in order to create the 
% right update queue}}
%\If{$|Y(c)| > 1$}{\Return{} \false}
%\If{\relation{c} \upshape is (==) }{\Return{} \true}
%\If{$c_j$ \upshape is not a linear equality}{
%  \Return{} \false}
%\int coeff = \coeff{c,v}\;
%\upshape in $\bigcup\limits_{o \in O} f_o(\vec{v})$}{	
%\tcp{Find the best variable to define}
\var $bestVariable$ = NULL\;
\int numberOfTies = 0 \;
 \ForEach{$x_i$ \upshape in $X(c_j)$ }{
%   \If{$x_i$ \upshape is fixed or defined}{
%     \continue
%   }
  \tcp{Break ties}
 % bestVariable = \decideBest{$x_i$,bestVariable} \;
  \If{\defi{$x_i$} $<$ \defi{$bestVariable$}}{
  \tcp{Choose the variable that helps define fewest invariants}
    $bestVariable$ = $x_i$ \;
    numberOfTies = 0 \;
%    \continue
  }
   \ElseIf{\defi{$x_i$} $==$ \defi{$bestVariable$}}{  
%     \If{$|D(x_i)|$ $>$ $|D(bestVariable)|$}{
%       \tcp{Choose the variable with largest domain size}
%       $bestVariable$ = $x_i$ \;
%       numberOfTies = 0 \;
%  %     \continue
%     } 
%     \ElseIf{$|D(x_i)|$ $==$ $|D(bestVariable)|$}{  
      \If{$|deg(x_i)|$ $<$ $|deg(bestVariable)|$}{
	\tcp{Choose the variable with lowest degree} \;
	$bestVariable$ = $x_i$ \;
	numberOfTies = 0 \;
%	\continue
      } 
      \ElseIf{$|deg(x_i)|$ $==$ $|deg(bestVariable)|$}{
	\tcp{Fair random} %\boste{Should this be further explained? (after alg)} 
	numberOfTies++ \;
	\If{Random(0,numberOfTies) == 0}{
	  $bestVariable$ = $x_i$ \;
	}
      }
%     }
   }
}

\If{ bestVariable $\neq $ \upshape NULL}{
  \makeOneway{\Var bestVariable} \;
 % \Return{} \true
}

%\Return{} \false \;
%\Return{} \true\;

\caption{Linear - canBeMadeOneway()} \label{algo_checkoneway} 
% \caption{Test if a constraint $c$ can define a variable $x$ } \label{algo_checkoneway}
\end{algorithm} \noindent
\DecMargin{1em} \\
For each unit \class{Linear} constraint an independent variable is found if possible. If there is more 
than one eligible variable the best variable among those is found. The first tiebreaker is the number of oneway 
constraints the variable participates in (helps define other variables). The next tiebreaker is the number of 
constraints the variables participate in. If none of the tiebreakers can be used, a fair random is used, such that the 
probability is equal for all variables whose ties could not be broken. \\ 
Once the best variable is found, if any, algorithm \ref{algo_makeoneway} \makeOneway is called. \\
%\begin{itemize}
% \item Highest search priority or highest domain size
% \item Lowest size of $C(x_i)$
% \item Fair random 
%\end{itemize}
%The variables that a constraint $c$ applies to is the scope $V(c)$. The constraints are of the type \class{Linear} and 
%a constraint $c$ have a right hand side $B(c)$. \\
\IncMargin{1em}
\begin{algorithm}[H]
\algdata
\Input{\Var $x_i$, \cons $c_j$}
\Output{A new invariant $y$ defined by a oneway constraint}
\BlankLine
set $Q = \emptyset$ \tcp*[r]{new coefficient set}
set $U = \emptyset$ \tcp*[r]{new variable set}
\tcp{Move other variables to right hand side and update their cofficient}
\ForEach{$x_k$ \upshape in $X(c_j)\setminus {x_i}$}{
  $c'_{kj} = -\frac{c_{kj}}{c_{ij}}$ \;   
  $Q = Q\cup c'_{kj}$ \;
  $U = Q\cup x_k$ \;
}
\tcp{Isolate $x_i$ }
\int $b' = \frac{b(c_j)}{c_{ij}}$ \; 
\textbf{invariant} $y = $ \Sum($U$,$Q$,$b'$)\; \tcp{Invariant whose value is defined by the other variables and a 
constant}
%$G$ = $G \cup \{inv\}$\; 
%$G$ = $G \setminus \{c_j\}$\; 
%$G$ = $G \setminus \{x_i\}$\;
  %\boste{Maybe remove $c$ from $G$}
  %\Return{\Sum{$V(c)$,$A(c)$,b}}

%\int $coef = A_{c,x}$\;
%$Q = A_c \backslash \{A_{c,x}\}$\;
%$U = V(c) \backslash \{x\}$\;
%\ForEach{$A_{c,x}$ \upshape in $Q$}{
%  $Q_{c,x} = A(c,x) \cdot \frac{-1}{coef}$
%}
%\int $b = B(c)$ \;
%\If{relation{c} \upshape is (==) }{
%  Invariant $c' = $ \Sum{$U$,$Q$,b}\;
%  $G$ = $G \cup c'$\; 
%  \boste{Maybe remove $c$ from $G$}
%  %\Return{\Sum{$V(c)$,$A(c)$,b}}
%}
%\Else{
%  Invariant $c' = $ \Sum{$U$,$Q$,b}\;
%  Invariant $c'' = $ \Max{c',lb(x)}\;
%  $G$ = $G \cup c'$\; 
%  $G$ = $G \cup c''$\; 
%  \boste{Maybe remove $c$ from $G$}
%  %\Return{\Max{inv,b}}
%}
 \caption{Linear - makeOneway(\textsf{Variable} $x_i$)} \label{algo_makeoneway}
\end{algorithm}\DecMargin{1em} \noindent
The algorithm transforms the constraint $c_j$ into a oneway constraint defining an invariant. The dependency directed 
graph $G$ is updated by adding the new invariant $y$ and removing the constraint $c_j$ and variable $x_i$. \\
The value of the invariant must always be within its domain, corresponding to domain of the variable. It is not allowed 
to change value of one of the variable such that the invariants value is not within its domain.
%\IncMargin{1em}
%\begin{algorithm}[H]
%\SetKwData{Oneway}{oneway}
%\SetKwFunction{makeOneway}{makeOneway}
%\SetKwFunction{Next}{next}\SetKwFunction{Constraints}{Constraints}\SetKwFunction{Remove}{remove}
%\SetKwFunction{canBeMadeOneway}{canBeMadeOneway}
%\algdata 
%\Input{A set $C$ of functional constraints sorted by increasing arity}
%\Output{A model better suited for local search}
%\BlankLine
%%\emph{special treatment of the first line}\;
%\Bool $change = $ \true\;
%\While{$C \neq \emptyset$ \textbf{and} $change$}{
%  $change = $\false \;
%  %\ForEach{$x \in X$}{
%  %  \upshape select \Var $x$ \upshape from $X$\;
%    \ForEach{\Con $c$ \upshape in $C$}{
%      \Bool $flag = $ \canBeMadeOneway{c}\;  
%      \If{$flag$}{
%	\makeOneway{c,x}\;
%	Remove $x$ from $X$\;
%	$change = \true$\;
%	\bre\;
%      }
%    }
%  }
%}
%\caption{Defining integer variables by one-way constraints}\label{makefunc}
%\end{algorithm}\DecMargin{1em}
%\noindent
%The following Two algorithms checks if the \cons $c$ can be transformed into a oneway constraint 
%and transforms $c_j$ into a one-way constraint defining $x$. 



%The algorithm creates invariants that defines variables $x_i \in X$ by one-way constraints. It uses two other 
%algorithms \canBeMadeOneway{$c_j$} and \makeOneway{$c_j$,$x_i$}. The first algorithm \\ 
%The complexity of algorithm \ref{algo_defintvar} depends on the complexity of the two other algorithms but for 
%simplicity let us assume they do not contribute for now. \\ 
%Let $\alpha_{max}$ be the largest arity among all constraints in $C$ and $n$ be the number of decision variables in 
%the 
%input set. The size of $X$ has decrease by at least one each time we pass line 3 except for the first time. Hence line 
%3 
%is passed at most $n$ times. Then the complexity of algorithm \ref{algo_defintvar} is $O(\alpha_{max} n^2)$. \\ 
%\medskip
%The coefficient of a variable $x_i$ in constraint $c_j$ is denoted $a_{ij}$. Let $\mathcal{F} = \{f_1,f_2,\dots 
%,f_k\}$ 
%be the family of objective functions \boste{Think we should discuss this Tuesday} and the coefficient of variable 
%$x_j$ in $f_k$ be $a_{kj}$. \boste{Maybe call it 
%evaluation functions. Does not make sense since $a_{34}$ refers both to constraint and obj. func} \\  


