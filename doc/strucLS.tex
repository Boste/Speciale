Once an initial solution to the constraint satisfaction optimization problem (COP) has been found, the model 
is transformed to create a model suited for local search. Two new datastructures are introduced 
in this section, dependency directed graph in subsection \ref{sec_ddg} and propagation queue in subsection 
\ref{sec_propaqueue}. 
\\
The dependency directed graph is used to handle dependencies of invariants when a 
variables value is changed. A propagation queue $q_i$ is created for each variable $x_i$ that gives an ordering 
of the 
invariants reachable from  $x_i$ in the dependency directed graph. \\
%When a variable $x$ changes value other variables dependent on the value of $x$ will need to be updated. To 
%update those variables and invariants a directed graph $G=(V,A)$ is made, called dependency directed graph, 
%\emph{DDG}. 
%The vertices $V$ either represent a variable, an invariant or a constraint. \\  
%When one or more variables change value they propagate their change to the invariants pointed in $G$ that might point 
%to other invariants and propagate their changes to them. We only want to visit each vertex in the graph $G$ at most 
%one 
%time when making changes to on or more variables to increase performance. For each variable used in 
%the local search a \emph{propagation queue} $q$ is made. A propagation queue is the order of which invariant to update 
%when the associated variable changes value. \boste{Not quite happy about the description here} \medskip \\
The model is simplified by defining some of the variables. This is done by transforming the functional constraints into 
oneway constraints, using the algorithms implemented in the respective constraints. When a variable is defined by a 
oneway constraint it is transformed into an invariant since its value is dependent on other variables and/or invariants. 
\\ 
The only constraint implemented, \class{Linear}, is used as an example for creating oneway constraints. 
 