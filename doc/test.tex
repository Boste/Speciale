\subsection{Parameters of Gecodes DFS Search Engine}
The search engine was briefly described in section \ref{sec_gecode} and the \class{Multistop} object can be used to 
stop the search for a solution. The search can be stopped based on three parameters, time, number of fails, and number 
of nodes explored. The number of nodes explored is highly correlated to the instance size and will not be tested. All 
the instances has been tested with a time limit of 100 seconds and random seed 60. This is done to get an indication 
of the time Gecode use and the number of failed spaces before Gecode finds a solution, if any. \\ 
The goal of the test is to find a parameter for time and number of fails such that the search can be stopped early if 
it is likely it will not find a solution. The worst case is Gecode never finds a solution hence the time used for 
searching has not be useful. \\
\begin{figure}[!h]
\centering
\includegraphics[width=\linewidth]{../R/GecodeTime} \caption{The seconds are on a logarithmic 
scale.}\label{fig_gecodetime}
% \end{center} 
\end{figure}\noindent
The result of the time is shown in figure \ref{fig_gecodetime} on the next page. The dots are blue if a solution has 
been found and red if Gecode did not find a solution. In all the instances where a solution was found it was found 
within 10 seconds. Based on this the time Gecode search for a solution is reduced to 10 seconds. \\ 
\begin{figure}[!h]
\centering
\includegraphics[width=\linewidth]{../R/GecodeFail} \caption{The number of failed spaces are on a logarithmic 
scale.}\label{fig_gecodefail}
% \end{center} 
\end{figure}\noindent
The number of failed spaces can be seen in figure \ref{fig_gecodefail} on the page after the time test. \\ 
In almost all of the instances where a solution was found, zero spaces where failed. The only exception is 
``neos-1440225'' that reported 19118 spaces to be failed. One of the instances where solution was not found reports 
that zero spaces were failed but otherwise they all report some failed spaces. Based on this test the number of failed 
spaces tolerated is set to 1. This will exclude one initial solution but save time on a lot of the others. \\ 


