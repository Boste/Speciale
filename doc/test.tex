\subsection{Parameters of Gecodes DFS Search Engine}
The search engine was briefly described in section \ref{sec_gecode} and the \class{Multistop} object can be used to 
stop the search for a solution. The search can be stopped based on three parameters, time, number of fails, and number 
of nodes explored. The number of nodes explored is highly correlated to the instance size and will not be tested. All 
the instances has been tested with a time limit of 100 seconds and random seed 60. This is done to get an indication 
of the time Gecode use and the number of failed spaces before Gecode finds a solution, if any. \\ 
The goal of the test is to find a parameter for time and number of fails such that the search can be stopped early if 
it is likely it will not find a solution. The worst case is Gecode never finds a solution hence the time used for 
searching has not be useful. \\
\begin{figure}[!h]
\centering
\includegraphics[width=\linewidth]{../R/GecodeTime} \caption{The seconds are on a logarithmic 
scale.}\label{fig_gecodetime}
% \end{center} 
\end{figure}\noindent
The result of the time is shown in figure \ref{fig_gecodetime} on the next page. The dots are blue if a solution has 
been found and red if Gecode did not find a solution. In all the instances where a solution was found it was found 
within 10 seconds. Based on this the time Gecode search for a solution is reduced to 10 seconds. \\ 
\begin{figure}[!h]
\centering
\includegraphics[width=\linewidth]{../R/GecodeFail} \caption{The number of failed spaces are on a logarithmic 
scale.}\label{fig_gecodefail}
% \end{center} 
\end{figure}\noindent
The number of failed spaces can be seen in figure \ref{fig_gecodefail} on the page after the time test. \\ 
In almost all of the instances where a solution was found, zero spaces where failed. The only exception is 
``neos-1440225'' that reported 19118 spaces to be failed. One of the instances where solution was not found reports 
that zero spaces were failed but otherwise they all report some failed spaces. Based on this test the number of failed 
spaces tolerated is set to 1. This will exclude one initial solution but save time on a lot of the others. \\ 
There are other parameters for Gecode that could have been tested is the different ways for Gecode to branch. \\ 
\phantom{p. 1}
\clearpage
%\thispagestyle{empty}
%%\phantom{p. 2}
%\clearpage
\subsection{Defining variables by Oneway Constraints}
If the mdeol contains functional constraints we try to define one variable for each functional constraint to reduce the 
search space. The effect has been test on the instances where functional constraints are present. They have been tested 
by algorithm 1, Tabu search with \class{ConflictOnlyNE} and tabu search with \class{FlipNeigborhood}, with a 
total run time of 120 seconds. The seed that was used during the test was $42$. \\ 
The resulting number of violations are shown in figure \ref{fig_oneway}. To see if defining variable 
by oneway has an impact on the result it has been analysed with Wilcoxon signed-rank test. The p-value of the test is 
0.8888 hence it is not statistical significant that one result was better than the other. \\ 
The two cases where the number of violations are zero for both, the objective function is zero as well. 
\begin{figure}[!h]
\centering
\includegraphics[width=\linewidth]{../R/oneway} \caption{Sorted by number of defined variables in 
increasing order.}\label{fig_oneway}
% \end{center} 
\end{figure}\noindent
\
\subsection{Using Gecode as Construction Heuristic}
Gecode set limits on the design but it might provide a useful preprocessing and initial solution as well. We test 
the effect of using Gecode to find and initial solution versus a random assignment to the variables. In both cases a 
first improvement is used until local optima just like describe in section \ref{sec_local}. Two test on 46 instances 
has been made to test the effect of Gecode. \\ 
The first test algorithm 2, Iterated local search, is used until a total time limit of 30 seconds, with a random seed 
$42$. The limit is chosen to make sure local search has started but not been running for very long. The violation of 
the result is shown in figure \ref{fig_gecodels}. 
\begin{figure}[!h]
\centering
\includegraphics[width=\linewidth]{../R/gecodels} \caption{Violation after 30 seconds with 
algorithm 1, with and without Gecode}\label{fig_gecodels}
% \end{center} 
\end{figure}\noindent
There is barely a difference between the two results and a Wilcoxon signed-rank test gives a p-value of 1, hence they 
are not significantly different. In order to look closer at the difference we look at the objective to see if there is 
a difference there. In order to visualize the data a ratio is created for each result, since the objective value span
from -206179 to 3810000 in different instances. 
\begin{equation}
 ratio = \frac{obj.val1}{obj.val1+obj.val2}
\end{equation}
$obj.val1$ is the objective value of the result without Gecode and $obj.val2$ is the result with gecode. This can only 
be done if the denominator is not zero and the values do not have opposite signs. In only two instances this was the 
case and they have been left out in the visualization. The two instances are ``neos808444'' and ``neos-849702'' and 
both have objective value zero. If the resulting ratio is 0.5 then the objective value is the same in the test. The 
result is shown in figure \ref{fig_gecodelsobj} and a Wilcoxon test gives a p-value of 0.8379. 
\begin{figure}[!h]
\centering
\includegraphics[width=\linewidth]{../R/gecodelsobj} \caption{Violation after 30 seconds with 
algorithm 1, with and without Gecode}\label{fig_gecodelsobj}
% \end{center} 
\end{figure}\noindent
The figure show they give an equal objective value most of the time. This could be because the local search 
evened out the difference, hence the next text we leave out the local search. \medskip \\
The test will be Gecode and first improvement versus random assignment and first improvement, in both cases until a 
local optima has been found. In this case the run with Gecode proves to be significant better than without Gecode. 
Wilcoxon test gives a p-value of 0.01894. The result of the test is shown in figure \ref{fig_gecodenols}. \\ 
\begin{figure}[!h]
\centering
\includegraphics[width=\linewidth]{../R/gecodenols} \caption{Violation after 30 seconds with 
algorithm 1, with and without Gecode}\label{fig_gecodenols}
% \end{center} 
\end{figure}\noindent
Based on the last test we keep using Gecode for further test, though the time usage is of interest as well. The time 
use is plotted in figure \ref{fig_gecodetime} and it shows Gecode uses more time which is also confirmed by a Wilcoxon 
test with a p-value of 0.005979. 
\begin{figure}[!h]
\centering
\includegraphics[width=\linewidth]{../R/gecodetime} \caption{Violation after 30 seconds with 
algorithm 1, with and without Gecode}\label{fig_gecodetime}
% \end{center} 
\end{figure}\noindent
\newpage
\subsection{Testing the Algorithms}






