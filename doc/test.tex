\subsection{Parameters of Gecodes DFS Search Engine} \label{sub_gectest}
The search engine was briefly described in section \ref{sec_gecode} and the \class{Multistop} object can be used to 
stop the search for a solution. The search can be stopped based on three parameters, time, number of fails, and number 
of nodes explored. The number of nodes explored is highly correlated to the instance size and will not be tested. All 
the instances has been tested with a time limit of 100 seconds and random seed 60. This is done to get an indication 
of the time Gecode use and the number of failed spaces before Gecode finds a solution, if any. \\ 
The goal of the test is to find a parameter for time and number of fails such that the search can be stopped early if 
it is likely it will not find a solution. The worst case is Gecode never finds a solution hence the time used for 
searching has not be useful. \\
\begin{figure}[!h]
\centering
\includegraphics[width=\linewidth]{../R/GecodeTime} \caption{The seconds are on a logarithmic 
scale.}\label{fig_gecodetime}
% \end{center} 
\end{figure}\noindent
The result of the time is shown in figure \ref{fig_gecodetime} on the next page. The dots are blue if a solution has 
been found and red if Gecode did not find a solution. In all the instances where a solution was found it was found 
within 10 seconds. Based on this the time Gecode search for a solution is reduced to 10 seconds. \\ 
\begin{figure}[!h]
\centering
\includegraphics[width=\linewidth]{../R/GecodeFail} \caption{The number of failed spaces are on a logarithmic 
scale.}\label{fig_gecodefail}
% \end{center} 
\end{figure}\noindent
The number of failed spaces can be seen in figure \ref{fig_gecodefail} on the page after the time test. \\ 
In almost all of the instances where a solution was found, zero spaces where failed. The only exception is 
``neos-1440225'' that reported 19118 spaces to be failed. One of the instances where solution was not found reports 
that zero spaces were failed but otherwise they all report some failed spaces. Based on this test the number of failed 
spaces tolerated is set to 1. This will exclude one initial solution but save time on a lot of the others. \\ 
There are other parameters for Gecode that could have been tested is the different ways for Gecode to branch. \\ 
\phantom{p. 1}
\clearpage
%\thispagestyle{empty}
%%\phantom{p. 2}
%\clearpage
\subsection{Defining variables by Oneway Constraints}
If the mdeol contains functional constraints we try to define one variable for each functional constraint to reduce the 
search space. The effect has been test on the instances where functional constraints are present. They have been tested 
by algorithm 1, Tabu search with \class{ConflictOnlyNE} and tabu search with \class{FlipNeigborhood}, with a 
total run time of 120 seconds. The seed that was used during the test was $42$. \\ 
The resulting number of violations are shown in figure \ref{fig_onewaytest}. To see if defining variable 
by oneway has an impact on the result it has been analysed with Wilcoxon signed-rank test. The p-value of the test is 
0.8888 hence it is not statistical significant that one result was better than the other. \\ 
The two cases where the number of violations are zero for both, the objective function is zero as well. 
\begin{figure}[!h]
\centering
\includegraphics[width=\linewidth]{../R/oneway} \caption{Sorted by number of defined variables in 
increasing order.}\label{fig_onewaytest}
% \end{center} 
\end{figure}\noindent
\
\subsection{Using Gecode as Construction Heuristic}
Gecode set limits on the design but it might provide a useful preprocessing and initial solution as well. We test 
the effect of using Gecode to find and initial solution versus a random assignment to the variables. In both cases a 
first improvement is used until local optima just like describe in section \ref{sec_local}. Two test on 46 instances 
has been made to test the effect of Gecode. \\ 
The first test algorithm 2, Iterated local search, is used until a total time limit of 30 seconds, with a random seed 
$42$. The limit is chosen to make sure local search has started but not been running for very long. The violation of 
the result is shown in figure \ref{fig_gecodels}. 
\begin{figure}[!h]
\centering
\includegraphics[width=\linewidth]{../R/gecodels} \caption{Violation after 30 seconds with 
algorithm 1, with and without Gecode}\label{fig_gecodels}
% \end{center} 
\end{figure}\noindent
There is barely a difference between the two results and a Wilcoxon signed-rank test gives a p-value of 1, hence they 
are not significantly different. In order to look closer at the difference we look at the objective to see if there is 
a difference there. In order to visualize the data a ratio is created for each result, since the objective value span
from -206179 to 3810000 in different instances. 
\begin{equation}
 ratio = \frac{obj.val1}{obj.val1+obj.val2}
\end{equation}
The objective value $obj.val1$ is the result without Gecode and $obj.val2$ is the result with gecode. This can only 
be done if the denominator is not zero and the values do not have opposite signs. In only two instances this was the 
case and they have been left out in the visualization. The two instances are ``neos808444'' and ``neos-849702'' and 
both have objective value zero. If the resulting ratio is 0.5 then the objective value is the same in the test. The 
result is shown in figure \ref{fig_gecodelsobj} and a Wilcoxon test gives a p-value of 0.8379. 
\begin{figure}[!h]
\centering
\includegraphics[width=\linewidth]{../R/gecodelsobj} \caption{Violation after 30 seconds with 
algorithm 1, with and without Gecode}\label{fig_gecodelsobj}
% \end{center} 
\end{figure}\noindent
The figure show they give an equal objective value most of the time. This could be because the local search 
evened out the difference, hence the next test we leave out the local search. \medskip \\
The test will be Gecode and first improvement versus random assignment and first improvement, in both cases until a 
local optima has been found. In this case the run with Gecode proves to be significant better than without Gecode. 
Wilcoxon test gives a p-value of 0.01894. The result of the test is shown in figure \ref{fig_gecodenols}. \\ 
\begin{figure}[!h]
\centering
\includegraphics[width=0.9\linewidth]{../R/gecodenols} \caption{Violation after 30 seconds with 
algorithm 1, with and without Gecode}\label{fig_gecodenols}
% \end{center} 
\end{figure}\noindent
Based on the last test we keep using Gecode for further test, though the time usage is of interest as well. The time 
use is plotted in figure \ref{fig_gecodetime} and it shows Gecode sometimes uses more time. The Wilcoxon test gives a 
p-value of 0.005979 so we can conclude the time use is significantly different. The red line indicate which of the runs 
that uses the most time. 
\begin{figure}[!h]
\centering
\includegraphics[width=0.9\linewidth]{../R/gecodetime} \caption{Violation after 30 seconds with 
algorithm 1, with and without Gecode}\label{fig_gecodetime}
% \end{center} 
\end{figure}\noindent
\newpage
\phantom{p. 2}
\clearpage
\subsection{Testing the Algorithms}
Algorithm one is using tabu search only considering independent variables that are in a conflicting constraint as long 
the the current solution is infeasible. When the solution is feasible it considers all independent variables if there 
is less than 5000 otherwise it selects expected 5000 at random before choosing the best. \\ 
Algorithm two is an iterated local search, first improvement and random walk. The third algorithm uses a minimum 
conflict heuristic as long as the current solution is infeasible. When feasible it uses the same as algorithm one. \\ 
They have been tested on all 64 instances and will primarily be compared based on the number of violations. The test 
time used is 120 seconds and the random seed is 42. Based on the previous test Gecode is used with the parameters found 
in subsection \ref{sub_gectest}. We try to define as many variables as possible as oneway constraints. \\ 
To illustrate the data one instance has been left out to make the other data more visible. Instead the data for that 
instance, ``netdiversion'', is shown in table \ref{tab_netd}. The data show that algorithm three is by far the best 
one. The minimum conflict heuristic is very effective but after 77 seconds it does not improve the solution anymore. \\ 
The result for the other instances is illustrated in figure \ref{fig_algs} with a logarithmic scale on the number of 
violations. Even though algorithm three performed much better than the other on ``netdiversion'' it does perform much 
worse than the others on ``neos-1337307''. From the graph it looks like algorithm two performs a bit worse than the 
others. The algorithms has not been fine tuned but the parameters are based on a few initial test and what is found in 
other literature. The parameters have significant impact on the performance of an algorithm. \\ 

\begin{table}[b]
\centering
\begin{tabular}{|r|r|r|r|}
\hline
            & objective    & violation & iterations \\ \hline
Algorithm 1 & 5713042 & 3112      & 685        \\ \hline
Algorithm 2 & 5712891 & 3470      & 1014       \\ \hline
Algorithm 3 & 5714768 & 20        & 9773710    \\ \hline
\end{tabular}
\caption{Result on instance ``netdiversion''.}
\label{tab_netd}
\end{table}


\begin{figure}[!h]
\centerline{
\includegraphics[width=1.5\linewidth]{../R/algs}} \caption{Violation after 30 seconds with 
algorithm 1, with and without Gecode}\label{fig_algs}
\end{figure}\noindent
\newpage\noindent
The algorithms have been ranked on each of the 64 instances in order to find the best algorithm overall. The result is 
shown in figure \ref{fig_boxplot} and the ranks are shown in box plots. The value they have been ranked by is 
$1.000.000\cdot violations + objective$ $value$ in order to make violation more important than the objective value. The 
algorithms have been given the rank 1,2, or 3 for each instance. In case of a tie they are given the average, i.e. they 
two are tied for second place they are given 2.5 each. \\
\begin{figure}[!h]
\centering
\includegraphics[width=\linewidth]{../R/algbox} \caption{Boxplots over the ranking of the 
algorithms.}\label{fig_boxplot}
% \end{center} 
\end{figure}\noindent
From the plot it looks like algorithm three is the best performing wining in at least 25 \% of the instances. Algorithm 
two did perform worst in at least 50 \% of the instances. \\ 
\subsection{Versus Gurobi}
The time limit of the tests are 120 seconds with random seed 42 and algorithm three. Gurobi has the same time limit and 
uses only one thread as this framework. \\ 
The tables on the next three pages show the final result when comparing to Gurobi. In the columns are instance name, 
number of variables, number of constraints, how many percentage Gecode solved for initial solution, and the time for 
finding initial solution. The next columns are the time used for finding a feasible solution, the value of the feasible 
solution, number of iterations with local search. The last four columns are final objective value, final number of 
violation, Gurobis solution value and the optimal value. If a feasible solution is not found the cells are given NA. 
If a optimal solution is not know, then the cell will contain NA. \\ 
Gurobi found a feasible solution in 52 instances, algorithm three found a feasible solution in 33 instances. In three 
instances algorithm three found a solution equal to Gurobi and in three instances it found a solution better than 
Gurobi. \\
Gurobi do better preprocessing than Gecode but Gurobi has been optimized towards MIPLIB instances. Based on the result 
the framework can match Gurobi on a few instances though Gurobi is far superior. 


\begin{sidewaystable}
\centering
\begin{tabular}{|l|r|r|r|r|r|r|r|r|r|r|r|}
\hline
Instance    & \#Vars & \#Cons & Gecode Sol. & Init. Sol. Time & Feas.Time & Feas.Val & 
\#Iter & Final Obj & Final Viol & Gurobi & Opt \\ \hline
acc-tight4 & 1620 & 3285 & 0 & 0.010000 & NA & NA & 7883990 & 0 & 3 & 0 & 0\\ \hline 
acc-tight5 & 1339 & 3052 & 0 & 0.010000 & NA & NA & 7986211 & 0 & 10 & 0 & 0\\ \hline 
acc-tight6 & 1335 & 3047 & 0 & 0.010000 & NA & NA & 8008028 & 0 & 7 & 0 & 0\\ \hline 
air04 & 8904 & 823 & 25 & 0.800000 & NA & NA & 207078 & 76428 & 120  & 56137 & 56137 \\ \hline 
circ10-3 & 2700 & 42620 & 0 & 0.410000 & NA & NA & 243144 & 470 & 8 & NA & NA\\ \hline 
cov1075 & 120 & 637 & 100 & 0.010000 & 0.010000 & 42 & 99555 & 21 & 0 & 20 & 20\\ \hline 
datt256 & 262144 & 11077 & 0 & 24.550000 & NA & NA & 153905 & 258 & 101 & NA & NA\\ \hline 
ex1010-pi & 25200 & 1468 & 100 & 1.720000 & 1.720000 & 742 & 8189 & 449 & 0 & 244 & NA\\ \hline 
ex10 & 17680 & 69608 & 0 & 5.440000 & NA & NA & 39155 & 99 & 63 & 100 & 100\\ \hline 
ex9 & 10404 & 40962 & 0 & 1.870000 & NA & NA & 75425 & 81 & 42 & 81 & 81\\ \hline 
f2000 & 4000 & 10500 & 50 & 0.110000 & NA & NA & 10475255 & 1971 & 50 & NA & NA\\ \hline 
go19 & 441 & 441 & 100 & 0.000000 & 0.000000 & 147 & 527525 & 86 & 0 & 84 & 84\\ \hline 
hanoi5 & 3862 & 16399 & 50 & 0.120000 & NA & NA & 11871831 & 1934 & 20 & NA & 1931\\ \hline 
iis-100-0-cov & 100 & 3831 & 100 & 0.020000 & 0.020000 & 59 & 33278 & 29 & 0 & 29 & 29\\ \hline 
iis-bupa-cov & 345 & 4803 & 100 & 0.030000 & 0.030000 & 129 & 16437 & 38 & 0 & 36 & 36\\ \hline 
iis-pima-cov & 768 & 7201 & 100 & 0.070000 & 0.070000 & 214 & 6806 & 35 & 0 & 33 & 33\\ \hline 
m100n500k4r1 & 500 & 100 & 100 & 0.000000 & 0.000000 & 0 & 449501 & -23 & 0 & -24 & -25\\ \hline 
methanosarcina & 7930 & 14604 & 100 & 0.150000 & 0.150000 & 5046 & 9835 & 5046 & 0 & 2916 & NA\\ \hline 
macrophage & 2260 & 3164 & 100 & 0.030000 & 0.030000 & 609 & 35803 & 508 & 0 & 374 & 374\\ \hline 
mspp16 & 29280 & 561657 & 50 & 51.390000 & 145.52 & 451 & 0 & 451 & 0 & 418 & 363\\ \hline 
n3div36 & 22120 & 4484 & 100 & 4.540000 & 4.540000 & 480400 & 6160 & 480400 & 0 & 132800 & 130800\\ \hline 
n3seq24 & 119856 & 6044 & 0 & 34.360000 & 41.04 & 841200 & 1465 & 841200 & 0 & 52200 & 52200\\ \hline 


\end{tabular}
\caption{Table of first 22 instances.}
\label{tab1}
\end{sidewaystable}
\begin{sidewaystable}
\centering
\begin{tabular}{|l|r|r|r|r|r|r|r|r|r|r|r|}
\hline
Instance    & \#Vars & \#Cons & Gecode Sol. & Init. Sol. Time & Feas.Time & Feas.Val & 
\#Iter & Final Obj & Final Viol & Gurobi & Opt \\ \hline
neos-1109824 & 1520 & 28979 & 0 & 0.180000 & NA & NA & 70641 & 808 & 6 & 378 & 378\\ \hline 
neos-1337307 & 2840 & 5687 & 50 & 0.070000 & NA & NA & 694423 & -201273 & 427 & 18 & -202319\\ \hline 
neos-1440225 & 1285 & 330 & 0 & 0.020000 & NA & NA & 3484127 & 36 & 6 & 36 & 36\\ \hline 
neos-1616732 & 200 & 1999 & 100 & 0.000000 & 0.000000 & 176 & 277059 & 159 & 0 & 161 & 159 \\ \hline 
neos-1620770 & 792 & 9296 & 50 & 0.030000 & NA & NA & 11363096 & 18 & 1 & 9 & 9\\ \hline 
neos18 & 3312 & 11402 & 100 & 0.040000 & 0.040000 & 22 & 10196 & 22 & 0 & 16 & 16\\ \hline 
neos-631710 & 167056 & 169576 & 0 & 21.280000 & 24.03 & 555 & 9978 & 555  & 0 & 203 & 203  \\ \hline 
neos-777800 & 6400 & 479 & 50 & 0.230000 & 4.05 & -80 & 14719 & -80 & 0  & -80 & -80\\ \hline 
neos808444 & 19846 & 18329 & 50 & 1.490000 & NA & NA & 3933914 & 0 & 43 & 0 & 0\\ \hline 
neos-849702 & 1737 & 1041 & 25 & 0.030000 & NA & NA & 987431 & 0 & 17 & NA & 0\\ \hline 
neos-941313 & 167910 & 13189 & 0 & 29.270000 & NA & NA & 5668704 & 19010 & 10 & 9361 & 9361\\ \hline 
netdiversion & 129180 & 119589 & 0 & 23.980000 & NA & NA & 9773710 & 5714768 & 20 & 400 & 242\\ \hline 
ns1663818 & 124626 & 172017 & 0 & 44.190000 & NA & NA & 0 & 0 & 165 & NA & 86\\ \hline 
ns1688347 & 2685 & 4191 & 25 & 0.100000 & NA & NA & 111441 & 35 & 6 & 270 & 27\\ \hline 
ns1696083 & 7982 & 11063 & 0 & 0.610000 & NA & NA & 61680 & 53 & 4 & NA & 45\\ \hline 
ns1853823 & 213440 & 224526 & 100 & 3.270000 & 3.270000 & 504000 & 4041 & 504000 & 0 & 467000 & NA\\ \hline 
ns894236 & 9666 & 8218 & 50 & 0.090000 & NA & NA & 1415963 & 18 & 35 & NA & NA\\ \hline 
ns894244 & 21856 & 12129 & 25 & 0.860000 & NA & NA & 983731 & 16 & 50 & NA & 15  \\ \hline 
ns894786 & 27278 & 16794 & 50 & 0.380000 & NA & NA & 1390731 & 14 & 29 & NA & NA\\ \hline 
ns894788 & 3463 & 2279 & 50 & 0.030000 & NA & NA & 1556360 & 7 & 16 & NA & 7\\ \hline 
ns903616 & 21582 & 18052 & 25 & 0.900000 & NA & NA & 783520 & 22 & 41 & NA & NA\\ \hline 
\end{tabular}
\caption{Table of next 21 instances.}
\label{tab3}
\end{sidewaystable}
\begin{sidewaystable}
\centering

\begin{tabular}{|l|r|r|r|r|r|r|r|r|r|r|r|}
\hline
Instance    & \#Vars & \#Cons & Gecode Sol. & Init. Sol. Time & Feas.Time & Feas.Val & 
\#Iter & Final Obj & Final Viol & Gurobi & Opt \\ \hline
opm2-z10-s2 & 6250 & 160633 & 100 & 0.750000 & 0.750000 & 0 & 1040 & -9297 & 0 & -24373 & -33826\\ \hline 
opm2-z11-s8 & 8019 & 223082 & 100 & 0.900000 & 0.900000 & 0 & 1097 & -10921 & 0 & -31311 & -43485\\ \hline 
opm2-z12-s14 & 10800 & 319508 & 100 & 1.520000 & 1.520000 & 0 & 1088 & -14557 & 0 & -43598 & -64291\\ \hline 
opm2-z12-s7 & 10800 & 319508 & 100 & 1.500000 & 1.500000 & 0 & 1408 & -15398 & 0 & -43139 & -65514\\ \hline 
opm2-z7-s2 & 2023 & 31798 & 100 & 0.110000 & 0.110000 & 0 & 4144 & -3695 & 0 & -10280 & -10280\\ \hline 
pb-simp-no & 23848 & 1451912 & 100 & 5.300000 & 5.300000 & 97 & 523 & 97 & 0 & 140 & NA\\ \hline 
protfold & 1835 & 2112 & 25 & 0.040000 & NA & NA & 880685 & -14 & 3 & -19 & -31\\ \hline 
p6b & 462 & 5852 & 100 & 0.010000 & 0.010000 & 0 & 79578 & -59 & 0 & -63 & -63\\ \hline 
queens-30 & 900 & 960 & 100 & 0.090000 & 0.090000 & 0 & 3124 & -38 & 0 & -37 & -40\\ \hline 
ramos3 & 2187 & 2187 & 100 & 0.030000 & 0.030000 & 471 & 22563 & 258 & 0 & 255 & NA\\ \hline 
seymour & 1372 & 4944 & 100 & 0.030000 & 0.030000 & 547 & 7780 & 446 & 0 & 424 & 423\\ \hline 
sts405 & 405 & 27270 & 100 & 0.160000 & 0.160000 & 355 & 1786 & 347 & 0 & 347 & NA\\ \hline 
sts729 & 729 & 88452 & 100 & 0.490000 & 0.490000 & 661 & 253 & 652 & 0 & 651 & NA\\ \hline 
tanglegram1 & 34759 & 68342 & 100 & 2.320000 & 2.320000 & 7813 & 4899 & 7516 & 0 & 5182 & 5182\\ \hline 
tanglegram2 & 4714 & 8980 & 100 & 0.080000 & 0.080000 & 2140 & 7175 & 2009 & 0 & 443 & 443\\ \hline 
toll-like & 2883 & 4408 & 100 & 0.030000 & 0.030000 & 1155 & 57839 & 931 & 0 & 627 & 610  \\ \hline 
t1717 & 73885 & 551 & 0 & 30.290000 & NA & NA & 96904 & 416368 & 25 & 238901 & NA \\ \hline 
t1722 & 36630 & 338 & 50 & 7.540000 & NA & NA & 81574 & 315879 & 33 & 136540 & NA\\ \hline 
vpphard2 & 199999 & 198450 & 0 & 2.990000 & NA & NA & 189071 & 752 & 28 & 81 & 81\\ \hline 
vpphard & 51471 & 47280 & 0 & 1.380000 & NA & NA & 294746 & 328 & 5 & 9 & 5\\ \hline 
wnq-n100-mw & 10000 & 656900 & 50 & 4.370000 & 10.18 & 5987 & 781 & 2411 & 0 & 284 & 259 \\ \hline
\end{tabular}
\caption{Table of last 21 instances.}
\label{tab3}
\end{sidewaystable}


\newpage

