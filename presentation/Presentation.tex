%%%%%%%%%%%%%%%%%%%%%%%%%%%%%%%%%%%%%%%%%
% Beamer Presentation
% LaTeX Template
% Version 1.0 (10/11/12)
%
% This template has been downloaded from:
% http://www.LaTeXTemplates.com
%
% License:
% CC BY-NC-SA 3.0 (http://creativecommons.org/licenses/by-nc-sa/3.0/)
%
%%%%%%%%%%%%%%%%%%%%%%%%%%%%%%%%%%%%%%%%%

%----------------------------------------------------------------------------------------
%	PACKAGES AND THEMES
%----------------------------------------------------------------------------------------

\documentclass{beamer}

\mode<presentation> {

% The Beamer class comes with a number of default slide themes
% which change the colors and layouts of slides. Below this is a list
% of all the themes, uncomment each in turn to see what they look like.

%\usetheme{default}
%\usetheme{AnnArbor}
%\usetheme{Antibes}
%\usetheme{Bergen}
%\usetheme{Berkeley}
%\usetheme{Berlin}
%\usetheme{Boadilla}
%\usetheme{CambridgeUS}
%\usetheme{Copenhagen}
%\usetheme{Darmstadt}
%\usetheme{Dresden}
%\usetheme{Frankfurt}
%\usetheme{Goettingen}
%\usetheme{Hannover}
%\usetheme{Ilmenau}
%\usetheme{JuanLesPins}
%\usetheme{Luebeck}
%\usetheme{Madrid}
%\usetheme{Malmoe}
%\usetheme{Marburg}
%\usetheme{Montpellier}
%\usetheme{PaloAlto}
%\usetheme{Pittsburgh}
%\usetheme{Rochester}
%\usetheme{Singapore}
%\usetheme{Szeged}
\usetheme{Warsaw}

% As well as themes, the Beamer class has a number of color themes
% for any slide theme. Uncomment each of these in turn to see how it
% changes the colors of your current slide theme.

%\usecolortheme{albatross}
%\usecolortheme{beaver}
%\usecolortheme{beetle}
%\usecolortheme{crane}
%\usecolortheme{dolphin}
%\usecolortheme{dove}
%\usecolortheme{fly}
%\usecolortheme{lily}
%\usecolortheme{orchid}
%\usecolortheme{rose}
%\usecolortheme{seagull}
%\usecolortheme{seahorse}
%\usecolortheme{whale}
%\usecolortheme{wolverine}

%\setbeamertemplate{footline} % To remove the footer line in all slides uncomment this line
%\setbeamertemplate{footline}[page number] % To replace the footer line in all slides with a simple
%slide count uncomment this line

\setbeamertemplate{navigation symbols}{} % To remove the navigation symbols from the bottom of all
%slides uncomment this line
}
\expandafter\def\expandafter\insertshorttitle\expandafter{%
  \insertshorttitle\hfill%
  \insertframenumber\,/\,\inserttotalframenumber}

\usepackage{algorithm,algorithmic}
\usepackage{graphicx} % Allows including images
\usepackage{booktabs} % Allows the use of \toprule, \midrule and \bottomrule in tables
\usepackage{tikz}
\usepackage[utf8]{inputenc}
\usepackage[danish]{babel}
\usetikzlibrary{arrows,decorations.pathmorphing,backgrounds,positioning,fit,matrix} 
\tikzstyle{vertex}=[circle,fill=black!25,minimum size=15pt,inner sep=0pt]
\tikzstyle{selectedvertex} = [vertex, fill=red!24] 
\tikzstyle{edge} = [draw,thick,-] 
\tikzstyle{arc}= [draw,thick,->,shorten >=1pt,>=stealth'] 
\tikzstyle{arcl} = [draw,thick,->,shorten>=1pt,>=stealth',bend left=25] 
\tikzstyle{arcr} = [draw,thick,->,shorten >=1pt,>=stealth']
\tikzstyle{rpath}=[draw, thick,->,shorten >=1pt,>=stealth',red, opacity=0.4]
\tikzstyle{weight} = [font=\small] 
\tikzstyle{selected edge} = [draw,line width=5pt,-,red!50]
\tikzstyle{ignored edge} = [draw,line width=5pt,-,black!20] 
\newcommand*{\vpointer}{\vcenter{\hbox{\scalebox{2}{\Huge\pointer}}}}	
\def\Arrow{\raisebox{3\height}{\scalebox{3}{$\Rightarrow$}}}

%----------------------------------------------------------------------------------------
%	TITLE PAGE
%----------------------------------------------------------------------------------------

\title[]{Et Lokalsøgningssystem til at Løse Diskrete 
Optimeringsproblemer}% The short title appears at 
%the bottom of every slide, the full title is only on the title page

\author{Bo Stentebjerg-Hansen} % Your name
\institute[IMADA] % Your institution as it will appear on the bottom of every slide, may be
%shorthand %to save space
{
Syddansk Universitet \\ % Your institution for the title page
\medskip
Institut for Matematik og Datalogi
%\textit{IMADA} % Your email address
}
\date{\today} % Date, can be changed to a custom date

\begin{document}

\begin{frame}
\titlepage % Print the title page as the first slide
\end{frame}

\begin{frame}
\frametitle{Overview} % Table of contents slide, comment this block out to remove it
\tableofcontents % Throughout your presentation, if you choose to use \section{} and \subsection{}
%commands, these will automatically be printed on this slide as an overview of your presentation
\end{frame}

%----------------------------------------------------------------------------------------
%	PRESENTATION SLIDES
%----------------------------------------------------------------------------------------

%------------------------------------------------
\section{Introduktion} % Sections can be created in order to organize your presentation into
%discrete blocks, all sections and subsections are automatically printed in the table of contents as
%an overview of the talk
%------------------------------------------------
%\subsection{Subsection Example} % A subsection can be created just before a set of slides with a
%common theme to further break down your presentation into chunks

\begin{frame}
\frametitle{Introduktion}
\begin{itemize}[<+->]
\item 
\item 
$O(log^* n + \Delta)$ rounds, $\Delta$ is the highest vertex degree 
\end{itemize}




\end{frame}

%------------------------------------------------	
%\section{Definition af problemer}
\begin{frame}
\frametitle{Binære optimeringsproblemer}
\begin{align}
 \text{Minimize }\; &z =  \mathbf{c}^T\mathbf{x} \\ 
 \text{subject to } \; & \mathbf{A}\mathbf{x} \leq \mathbf{b} \\ 
 & \mathbf{x} \in \{0,1\}^n
\end{align} \noindent
$\mathbf{A}$ er en $m \times n$ matrice, $\mathbf{c}$ og $\mathbf{b}$ er $n$ dimensionale vectorer, alle tre består af 
heltal. $\mathbf{x}$ er en $n$ dimensional vector bestående af binære variable.   
\end{frame}

%------------------------------------------------

\begin{frame}
\frametitle{Eksempel}
% \begin{align}
%  \text{Minimize }\; z =  2&x_1 + x_2 + x_3\\ 
%  \text{subject to } \;  - &x_1 + 2x_2 \leq 1 \\
%  & x_1 +x_2 + x_3 = 2 \\
%  & \mathbf{x} \in \{0,1\}^3
% \end{align} \noindent
\begin{table}[]
\begin{center}
\label{my-label}
\begin{tabular}{llrcrlrl}
Minimize   & z = & $2x_1$        & +  & $x_2$       & + & $x_3$ &          \\
subject to &     & $-x_1$        & + & $2x_2$      &   &       & $\leq 1$  \\
           &     & $x_1$         & + & $x_2$       & + & $x_3$ & $=2$     
               
\end{tabular}
\end{center}
$\qquad \qquad $  $x_1,x_2,x_3 $  $\in$ $\{0,1\}$ 
\end{table}
En mulig løsning: \\
\vspace{0.2cm}
\begin{minipage}{0.47\linewidth}
 $x_1 = 1$  \\
 $x_2 = 0$ \\ 
 $x_3 = 1$ \\
\end{minipage}
\begin{minipage}{0.47\linewidth}

 $z = 2\cdot 1 + 1\cdot 0 + 1\cdot 1  =3$ \\
\end{minipage}

\end{frame}

%------------------------------------------------

\section{Løsningsmetoder}
\begin{frame}
\frametitle{Helttals programmering}
\begin{itemize}[<+->]  
\item Simplex metode.
\item Ligningsbaseret model.
\item Gurobi, GLPK, SCIP.
\end{itemize}
\begin{center}
\end{center}
\end{frame}

%------------------------------------------------

\begin{frame}
\frametitle{Constraint programming}
\begin{itemize}[<+->]
\item Bruger søgetræer til at finde en løsning.
\item Mere naturlig formulering af problemer.
\item bl.a. Gecode, prolog.
\end{itemize}
% \begin{center}
\end{frame}

\begin{frame}
\frametitle{Lokal søgning}
\begin{itemize}[<+->]
\item Undersøger mange små ændringer.
\item Kan ikke garentere optimalitet.
\item Ofte skrevet til et specifikt problem.
\end{itemize}
% \begin{center}
\end{frame}


\begin{frame}
\frametitle{Constraint programming med lokal søgning}
\begin{itemize}[<+->]
\item Formulering af problem som i Constraint programming. 
\item Genanvendelse af algorithmer.
\item Giver mulighed for at fokusere på modellering.
\item Solver fx Comet og OscaR.
\end{itemize}
% \begin{center}
\end{frame}


\begin{frame}
\frametitle{Det er projekt}
\begin{itemize}[<+->] 
\item Kombinere Gecode og lokal søgning.
\item Undersøger effekten af Gecode. 
\item Tester brugen af invarianter. 
\item Introducere en ny evaluerings metode.
\end{itemize}
\end{frame}


%------------------------------------------------
% \section{Finding a maximal matching}
%------------------------------------------------


%------------------------------------------------
\section{Opbygning af Systemet}

\begin{frame}
\frametitle{Overblik}
\begin{itemize}[<+->] 
\item Objekter, en kasse med værktøj og information. 
\item Brugerflade og to delt system. 
\end{itemize}
Her er et flot billed af formulering -> GPSolver -> GecodeEngine -> LocalSearcEngine. 

\end{frame}


\begin{frame}
\frametitle{Brugen af Gecode}
\begin{itemize}[<+->]
\pause 
\item Opret variable og begrænsninger.
\item Preprocessering af Gecode.
\item Oprettelse af søgningsstrategi. 
\item Finder måske en gyldig løsning.
\end{itemize}
\end{frame}

%------------------------------------------------
\begin{frame}
\frametitle{Transformation af Modellen}
\begin{itemize}[<+->]
\item 
\item 
\item 
\item 
\end{itemize}
\end{frame}

%------------------------------------------------

\begin{frame}
\frametitle{Maximal matching for $G$}
The maximal matching of $\mathcal{M} = M_1,\dots, M_\Delta$ can be computed. The
maximal matching $M_1$ of forest $F_1$ is computed and the vertices removed from $V$. Then the
maximal matching $M_2$ is computed and vertices removed and so on for all forests $F_3',\dots ,
F_\Delta'$. This gives a total of 3$\Delta$ rounds

\end{frame}

%------------------------------------------------


\begin{frame}
\frametitle{Algorithm}

\begin{algorithm}[H]
\begin{algorithmic}[1]
 {\footnotesize \STATE Compute forest decomposition $F_1 \dots F_\Delta$ 
 \STATE Make all edges directed from lowest ID to higher ID 
 \STATE Compute 3-coloring of each forest $F_i$ in parallel
 \STATE $\mathcal{M}\leftarrow \emptyset$
 \STATE $V' \leftarrow V$ 
\FOR{$i\leftarrow1$ to $\Delta$}
\FOR{$j\leftarrow1$ to $3$}
 \STATE let $c_j$ be the set of vertices colored $j$ in $F_i$
 \STATE Every $u \in c_j \cap V'$ selects one outgoing edge in $(V',E_i)$
 \STATE Let $M_j$ be the set of edges selected
 \STATE $\mathcal{M} \leftarrow \mathcal{M} \cup M_j$
 \STATE $V' \leftarrow V'\setminus \{u \mid u$ is matched in $M_j\}$ 
\ENDFOR
\ENDFOR }
\end{algorithmic}
\caption{pseudocode for maximal matching}
\end{algorithm}
\end{frame}

%------------------------------------------------
\section{Analysis}
\begin{frame}
\frametitle{Analysis}
\begin{itemize}
 \item First the graph $G$ is decomposed into at most $\Delta$ forests in constant time
\item In each forest the trees can be 3-colored in $O(log^* n)$ rounds in parallel
\item Computing the maximal matching in a tree takes three rounds
\item There are $\Delta$ forest hence the number of rounds for the matching is $O(\Delta)$
\item The total number of rounds need is $O(log^* n + \Delta)$
\end{itemize}
\end{frame}



%------------------------------------------------
\begin{frame}
 \frametitle{Questions?}
\begin{center}

{\Huge Questions?}
\end{center}
\end{frame}

\begin{frame}
 \frametitle{The End}
\begin{center}

{\Huge Thanks for your attention}
\end{center}
\end{frame}


%------------------------------------------------

% \begin{frame}[fragile] % Need to use the fragile option when verbatim is used in the slide
% \frametitle{Verbatim}
% 
% \end{frame}
%----------------------------------------------------------------------------------------

\end{document}